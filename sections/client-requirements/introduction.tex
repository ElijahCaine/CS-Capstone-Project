\section{Introduction}

\subsection{Purpose}
The purpose of this document is to present a detailed description of high performance XML/XSLT transformation server.
This document is intended for people who maintain server and the developers of the system. 
%%%%%%%%%%%%%%%%%%%%%%%%%%%%%%%%%%%%%%%%%%%%%%%%%%%%%%%%%%%%%%%%%%%%%%%%%%%%%%%%
% Delineate the purpose of the SRS
% Specify the intended audience of the SRS
%%%%%%%%%%%%%%%%%%%%%%%%%%%%%%%%%%%%%%%%%%%%%%%%%%%%%%%%%%%%%%%%%%%%%%%%%%%%%%%%

\subsection{Scope}
This software system will create a high performance XML/XSLT transformation server that can perform repetitive document transformations in a high-volume environment. 
XSLT is used to convert an XML document to another XML document, or other types of documents that can be recognized by the browser, such as HTML and XHTML. 
In normal, XSLT accomplishes this by converting each XML element to an (X)HTML element. With XSLT, you can add or remove elements and attributes from or to the output file. 
You can also rearrange elements, perform tests, and decide which elements to hide or display. 
%%%%%%%%%%%%%%%%%%%%%%%%%%%%%%%%%%%%%%%%%%%%%%%%%%%%%%%%%%%%%%%%%%%%%%%%%%%%%%%%
% 1. Identify the software product(s) to be produced by name (e.g., Host DBMS, Report Generator, etc.);
% 2. Explain what the software product(s) will, and, if necessary, will not do;
% 3. Describe the application of the software being specified, including relevant benefits, objectives, and goals;
% 4. Be consistent with similar statements in higher-level specifications (e.g., the system requirements specification), if they exist.
%%%%%%%%%%%%%%%%%%%%%%%%%%%%%%%%%%%%%%%%%%%%%%%%%%%%%%%%%%%%%%%%%%%%%%%%%%%%%%%%

\subsection{Definitions, acronyms, and abbreviations}
XML is a simple and standard way of exchanging raw data between computer programs.
It is not only easy to be written and read but also solve the application of information exchange between systems. 
There are two basic needs:
\begin{enumerate}
  \item Separate the data from the representation
  \item Transfer data between different applications
\end{enumerate}
In order to make the data easy for people to read and understand, we need to display information or pint out. For example, the data into an HTML file, a PDF file, or even a sound. Similarly, in order to adapt the data to different applications, a data format is converted to another data format, such as the demand format may be a text file, a SQL statement and an HTTP message. XSLT is used to achieve this conversion function of the language. XML is converted to HTML, XSLT is the most important function. 
XSLT: Extensible Stylesheet Language Transformations
XML: eXtensible Markup Language


%%%%%%%%%%%%%%%%%%%%%%%%%%%%%%%%%%%%%%%%%%%%%%%%%%%%%%%%%%%%%%%%%%%%%%%%%%%%%%%%
% Provide the definitions of all terms, acronyms, and abbreviations required to properly interpret the SRS.
% This information may be provided by reference to one or more appendixes in the SRS or by reference to other documents.
%%%%%%%%%%%%%%%%%%%%%%%%%%%%%%%%%%%%%%%%%%%%%%%%%%%%%%%%%%%%%%%%%%%%%%%%%%%%%%%%

\subsection{References}
IEEE. IEEE Std 830-1998 IEEE Recommended Practice for Software Requirements Specifications. IEEE Computer Society, 1998.

%%%%%%%%%%%%%%%%%%%%%%%%%%%%%%%%%%%%%%%%%%%%%%%%%%%%%%%%%%%%%%%%%%%%%%%%%%%%%%%%
% 1. Provide a complete list of all documents referenced elsewhere in the SRS;
% 2. Identify each document by title, report number (if applicable), date, and publishing organization;
% 3. Specify the sources from which the references can be obtained.
% 4. This information may be provided by reference to an appendix or to another document.
%%%%%%%%%%%%%%%%%%%%%%%%%%%%%%%%%%%%%%%%%%%%%%%%%%%%%%%%%%%%%%%%%%%%%%%%%%%%%%%%

\cite{xalan}
\cite{xerces}
\cite{icu}

\bibliographystyle{plain}
\bibliography{sections/references}

\subsection{Overview}
The next chapter, the Overall Description Section, of this document gives an overview of the
functionality of the product. It describes the informal requirements and is used to establish a context for the technical requirements specification in the next chapter.
The third chapter, Requirements Specification section, of this document is written primarily for the developers and describes in technical terms the details of the functionality of the product.
Both sections of the document describe the same software product in its entirety, but are intended for different audiences and thus use different language.


%%%%%%%%%%%%%%%%%%%%%%%%%%%%%%%%%%%%%%%%%%%%%%%%%%%%%%%%%%%%%%%%%%%%%%%%%%%%%%%%
% Describe what the rest of the SRS contains;
% Explain how the SRS is organized
%%%%%%%%%%%%%%%%%%%%%%%%%%%%%%%%%%%%%%%%%%%%%%%%%%%%%%%%%%%%%%%%%%%%%%%%%%%%%%%%
