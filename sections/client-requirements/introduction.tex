\section{Introduction}

\subsection{Purpose}
%%%%%%%%%%%%%%%%%%%%%%%%%%%%%%%%%%%%%%%%%%%%%%%%%%%%%%%%%%%%%%%%%%%%%%%%%%%%%%%%
% Delineate the purpose of the SRS
% Specify the intended audience of the SRS
%%%%%%%%%%%%%%%%%%%%%%%%%%%%%%%%%%%%%%%%%%%%%%%%%%%%%%%%%%%%%%%%%%%%%%%%%%%%%%%%

This document is intended to present a detailed description of the high performance XML/XSLT transformation server being developed by the Oregon State University CS Capstone Team ``XZES40''.
The intended audience for this document are the developers and sponsors of the project.

% The purpose of this document is to present a detailed description of high performance XML/XSLT transformation server.
% This document is intended for people who maintain server and the developers of the system.

\subsection{Scope}
%%%%%%%%%%%%%%%%%%%%%%%%%%%%%%%%%%%%%%%%%%%%%%%%%%%%%%%%%%%%%%%%%%%%%%%%%%%%%%%%
% 1. Identify the software product(s) to be produced by name (e.g., Host DBMS, Report Generator, etc.);
% 2. Explain what the software product(s) will, and, if necessary, will not do;
% 3. Describe the application of the software being specified, including relevant benefits, objectives, and goals;
% 4. Be consistent with similar statements in higher-level specifications (e.g., the system requirements specification), if they exist.
%%%%%%%%%%%%%%%%%%%%%%%%%%%%%%%%%%%%%%%%%%%%%%%%%%%%%%%%%%%%%%%%%%%%%%%%%%%%%%%%

The name of this software, for lack of a better one, will be XZES40-Transformer.

The core product being delivered is a high performance XML/XSLT transformation server.
This server will be able to perform repetitive document transformations quickly and efficiently by caching previously processed and compiled documents.
Time is saved by pulling from an in-memory cache of documents and their compiled state rather than downloading and compiling documents which have already been processed, as current systems tend to do.
XZES40-Transformer will also carry out transformations in parallel.
It will transfer documents to and from clients via the HTTP or HTTPS protocol.

The target platform for XZES40-Transformer will be Debian Linux 8 (``Jessie'').
The core product will be designed to allow the program to be ported easily from Linux to other operating systems like Windows and BSD.

In addition to the core server there will also be a command-line interface and web-interface developed to interact with the application, these will be called XZES-CLI and XZES-Web respectively.

% This software system will create a high performance XML/XSLT transformation server that can perform repetitive document transformations in a high-volume environment.
% XSLT is used to convert an XML document to another XML document, or other types of documents that can be recognized by the browser, such as HTML and XHTML.
% In normal, XSLT accomplishes this by converting each XML element to an (X)HTML element.
% With XSLT, you can add or remove elements and attributes from or to the output file.
% You can also rearrange elements, perform tests, and decide which elements to hide or display.

\subsection{Definitions, acronyms, and abbreviations}
%%%%%%%%%%%%%%%%%%%%%%%%%%%%%%%%%%%%%%%%%%%%%%%%%%%%%%%%%%%%%%%%%%%%%%%%%%%%%%%%
% Provide the definitions of all terms, acronyms, and abbreviations required to properly interpret the SRS.
% This information may be provided by reference to one or more appendixes in the SRS or by reference to other documents.
%%%%%%%%%%%%%%%%%%%%%%%%%%%%%%%%%%%%%%%%%%%%%%%%%%%%%%%%%%%%%%%%%%%%%%%%%%%%%%%%

Below is a list of acronyms and abbreviations used throughout the document:

\begin{itemize}
  \item Extensible Markup Language XML: The human-readable data format used and processed by our application.
  \item Extensible Stylesheet Markup Language (XSLT): The human-readable format used to transform documents in our application.
  \item Xerces-C \cite{xerces}: One library used to perform XML transformation in C.
  \item Xalan-C \cite{xalan}: One library used to XML transformation in C.
  \item ICU \cite{icu}: One library used to process UTF-8 character formated documents.
  \item Hypertext Transfer Procol (HTTP/HTTPS): The protocol over which XZES40-Transformer will interact with remote clients.
  \item HTTP Application Programming Interface (HTTP API): A standard way of communicating with a web application.
  \item Unified Resource Locator / Identifier (URL/URI): Addressable location of a resource over the internet (e.g., a website address).
  \item Debian 8 (``Jessie''): The target Linux-based operating system XZES40-Transformer will run on.
  \item Unicode Transmission Format 8 (UTF-8): The international standard for encoding text-based data.
  \item Apache Web-server: A Free and Open Source web-server.
  \item Common Gateway Interface Script (CGI Script): Server-side scripts that can run applications on client's behalf.
\end{itemize}
% XML is a simple and standard way of exchanging raw data between computer programs.
% It is not only easy to be written and read but also solve the application of information exchange between systems.
% There are two basic needs:
% 
% \begin{enumerate}
%   \item Separate the data from the representation
%   \item Transfer data between different applications
% \end{enumerate}
% 
% In order to make the data easy for people to read and understand, we need to display information or pint out.
% For example, the data into an HTML file, a PDF file, or even a sound.
% Similarly, in order to adapt the data to different applications, a data format is converted to another data format, such as the demand format may be a text file, a SQL statement and an HTTP message.
% XSLT is used to achieve this conversion function of the language.
% XML is converted to HTML, XSLT is the most important function.
% XSLT: Extensible Stylesheet Language Transformations
% XML: Extensible Markup Language

\clearpage

\subsection{References}
%%%%%%%%%%%%%%%%%%%%%%%%%%%%%%%%%%%%%%%%%%%%%%%%%%%%%%%%%%%%%%%%%%%%%%%%%%%%%%%%
% 1. Provide a complete list of all documents referenced elsewhere in the SRS;
% 2. Identify each document by title, report number (if applicable), date, and publishing organization;
% 3. Specify the sources from which the references can be obtained.
% 4. This information may be provided by reference to an appendix or to another document.
%%%%%%%%%%%%%%%%%%%%%%%%%%%%%%%%%%%%%%%%%%%%%%%%%%%%%%%%%%%%%%%%%%%%%%%%%%%%%%%%

% IEEE. IEEE Std 830-1998 IEEE Recommended Practice for Software Requirements Specifications. IEEE Computer Society, 1998.

\printbibliography

\clearpage

\subsection{Overview}
%%%%%%%%%%%%%%%%%%%%%%%%%%%%%%%%%%%%%%%%%%%%%%%%%%%%%%%%%%%%%%%%%%%%%%%%%%%%%%%%
% Describe what the rest of the SRS contains;
% Explain how the SRS is organized
%%%%%%%%%%%%%%%%%%%%%%%%%%%%%%%%%%%%%%%%%%%%%%%%%%%%%%%%%%%%%%%%%%%%%%%%%%%%%%%%

\tableofcontents

% The next chapter, the Overall Description Section, of this document gives an overview of the
% functionality of the product.
% It describes the informal requirements and is used to establish a context for the technical requirements specification in the next chapter.
% The third chapter, Requirements Specification section, of this document is written primarily for the developers and describes in technical terms the details of the functionality of the product.
% Both sections of the document describe the same software product in its entirety, but are intended for different audiences and thus use different language.
