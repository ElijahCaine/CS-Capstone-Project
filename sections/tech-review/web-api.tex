% Section Six: Web API
% Owned by Elijah C. Voigt
\section{Web API}

XZES40-Transformer will be accessible via a Web API.
This can be implemented a few different ways, but all of them must accomplish the same goal: people across the world to use the service over the web.

\subsection{Web application framework: kore.io}

One option for every web-accessible application is to write a web application which exposes your project to the world.
To this end a fairly simple library in development since 2013 called \textit{kore} seems like a fair candidate.
This would be used to handle incoming requests, call the XZES40-Transformer application, and return the transformed document.

This has some upsides and down-sides.
The application will no-doubt be fast and allow our application to be entirely self contained in the C code.
The application daemon which preserves our cache in memory could double as our web-application daemon.
The downside to this is that it will increase the size of our code-base, which increases the amount of code we have to maintain, and can easily be a resource sink-hole.

\subsection{Apache CGI script}

A solution proposed by our client was to use the Apache web-server's CGI interface to expose our application.
This would require writing a short script, probably in python, to handle the incoming and outgoing request.
The script would call our application and take the output file and send it back to the requested user.
\cite{apache-cgi-tutorial}

This has many benefits, chief among them being that it will cut down on development time.
Once we have a working command-line interface for local document transformation we can write a simple script to shell out to this, making it accessible to the world.

\subsection{???}

\subsection{Conclusion}

We will chose to use an Apache CGI script for our web API.
