% Section Eight: Debian Package
% Owned by Zixun Lu 
%\section{Debian Package}
%XZES40-Transformer target the Debian operating system. Our team will upload XZES-40.deb through libraries(Xerces, Xalan) to the Apache website. The users can directly download from the website and run in their Debian operation system. 
%\subsection{Binary packages}
%Binary packages, which contain executables, configuration files, man/info pages, copyright information, and other documentation. These packages are distributed in a Debian-specific archive format. They are usually characterized by having a '.deb' file extension. Binary packages can be unpacked using the Debian utility dpkg (possibly via a frontend like aptitude); details are given in its manual page.
%\subsection{Source packages}
%Source packages, which consist of a .dsc file describing the source package (including the names of the following files), a .orig.tar.gz file that contains the original unmodified source in gzip-compressed tar format and usually a .diff.gz file that contains the Debian-specific changes to the original source. The utility dpkg-source packs and unpacks Debian source archives; details are provided in its manual page. (The program apt-get can be used as a frontend for dpkg-source.)
%\subsection{}
\section{Debian & Centos Package}

\subsection{Options}


\subsection{Goals for use in design}

Our team will release Debian and Centos packages.
Users can directly download these from the website and use these packages to install XZES40-Transformer on their Debian and Centos operation system.

\subsection{Criteria being evaluated}

We may use Centos and Debian tools to build our package.
We can use Centos operation system to build Centos packages and use Debian operation system to build Debian packages.
Those tools are free to use.

\subsection{Comparison breakdown}

\begin{center}
  \begin{tabular}{ | l | p{10cm} |}
    \hline
    Technology & Description  \\ \hline
    Centos packaging tools \cite{centos-tool} &
    \begin{itemize}
      \item Use in Centos operate system 
      \item It is esay to use
      \item Free.
    \end{itemize}\\ \hline
    FPM \cite{fpm-home} &
    \begin{itemize}
      \item Translates packages from one format to another.
      \item Allows re-use of other system's packages.
      \item Free.
    \end{itemize}\\ \hline
    Debian packaging tools \cite{debian-tool} &
    \begin{itemize}
      \item Use in Debian operate system
      \item It is easy to use
      \item Free.
    \end{itemize}\\ \hline
  \end{tabular}
\end{center}

\subsection{Discussion}

The above tools are all valuable.
If we were going to just develop a Debian package we may only use the Debian tools, and the same goes for CentOS.
These tools are good at creating packages for those specific platforms, but since we intend to develop tools for mutiple platforms (Linux and BSD) using FPM to create cross-OS packages would be very convenient.

\subsection{Selection}

In the end we will use FPM to develop our packages becuase it makes life very convenient.
