% Section Five: Prallel Transformation 
% Owned by Zixun Lu 
\section{Parallel Document Transformation}
\subsection{Options}
\subsection{Goals for use in design}

XZES40-Transformer will be increased performance by using parrallel computation.
The parallel processing of the containment queries against an XML document utilizes parallel variants of the serial algorithm. 
There are two ways to do the prallel computation. One is Round-Robin distribution of index entries, the other is Hash-based distribution of index entries. 

\subsection{Criteria being evaluated}

XZES40-Transformer use the parallel computation method do the same processing and operating side by side. First, the entries of the fully-inverted index are distributed among the cluster nodes for processing. Second, the containment query is processed by the cluster nodes to generate the corresponding lists of index entries. Third, the elements of the generated lists are checked against one another to produce the result set. This method will be increased the efficiency and save time. 

\subsection{Comparison breakdown}

The first option is Round-Robin distribution of index entries which is the master node distributes the index entries uniformly over the slave nodes. The second option is Hash-based distribution of index entries which algorithm uses a hash function to partitions the index entries into a number of disjoint sets.

\begin{center}
    \begin{tabular}{ | l | p{10cm} |} 
    \hline
    Technology & Description    \\ \hline
    Round-Robin distribution of index entries 
    the number of index entries allocated to each node will be (1/n) * size of index-table, where n is the number of slave nodes in the      Beowulf cluster. It is important to note here that the index entries with the same term values might land into the same or different slave nodes.   \\ \hline
    Hash-based distribution of index entries
    This number is equal to that of the slave nodes. Each one of the disjoint sets is then assign to a different slave node for further processing. As a result, the number of index entries allocated to each node will roughly be (1/n) * size of index-table, where n is the number of slave nodes in the Beowulf cluster. Several variants do exist for this algorithm based on the entity that computes the hash function and the field in the index entry for which the hash function is computed. The entity that computes the hash function can be the master node or the slave ones. In the latter case, we assume that the complete inverted index is replicated across all of the slave nodes, and therefore, each one of these nodes can apply the hash function to its local copy and select those entries that maps to the slave node itself. This process eliminates the initial need to transfer the index terms across the network. This step is skipped if the index entries are already distributed among the slave nodes as result of processing an earlier query. \\ \hline
    \end{tabular}
\end{center}

\subsection{Discussion}

The Round-Robin method is suitable for queries which requires scanning the entire relation. 
The Hash method sequential scan can be done efficiently.

\subsection{Selection}

We will both use Round-Robin and Hash method. 

