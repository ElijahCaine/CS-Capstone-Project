% Section Five: Prallel Transformation 
% Owned by Zixun Lu 
\section{Parallel Document Transformation}
\subsection{Options}
\subsection{Goals for use in design}

XZES40-Transformer will be increased performance by using parrallel computation.
The parallel processing of the containment queries against an XML document utilizes parallel variants of the serial algorithm. 
There are two ways to do the prallel computation. 

\subsection{Criteria being evaluated}

XZES40-Transformer use the parallel computation method do the same processing and operating side by side. First, the entries of the fully-inverted index are distributed among the cluster nodes for processing. Second, the containment query is processed by the cluster nodes to generate the corresponding lists of index entries. Third, the elements of the generated lists are checked against one another to produce the result set. This method will be increased the efficiency and save time. 

\subsection{Comparison breakdown}

\begin{center}
  \begin{tabular}{ | l | p{10cm} | }
    \hline
    Technology & Description  \\ \hline
    POSIX Threads \cite{posix threads} &
    \begin{itemize}
      \item Defines a set of C types, functions and constants 
      \item Spawn concurrent units of processing
      \item Achieve big speedups, as all cores of CPU are used at the same time. 
      \end{itemize}\\ \hline
    OpenMP \cite{Openmp} &
    \begin{itemize}
      \item An API that implements a multi-thread, shared memory form of parallelism.
      \item Uses a set of compiler directives 
      \item Take care of many of low-level details
    \end{itemize}\\ \hline
    MPI \cite{mpi} &
    \begin{itemize}
      \item Core syntax and semantics of library
      \item Complexity, scope and control
      \item Manage allocation, communication, and synchronization of a set of processes 
    \end{itemize}\\ \hline
  \end{tabular}
\end{center}



\subsection{Discussion}

Pthreads is a standard for prgramming with threads, it can achieve big speedups, as all cores of your CPU are used at the same time.
OpenMP uses a set of compiler directives that are incorporated at compile-time to generate a multi-threaded version of your code.
MPI allows us to manage allocation, communication, and synchronization of a set of process.

\subsection{Selection}

We will use MSI because it is a high-level standard. 

