%%%%%%%%%%%%%%%%%%%%%%%%%%%%%%%%%%%%%%%%%%%%%%%%%%%%%%%%%%%%%%%%%%%%%%%%%%%%%%%%
% What did you learn from all this?
% One per team member.
% > What technical information did you learn?
% > What non-technical information did you learn?
% > What have you learned about project work?
% > What have you learned about project management?
% > What have you learned about working in teams?
% > If you could do it all over, what would you do differently?
% Be honest here -- no B.S.
%%%%%%%%%%%%%%%%%%%%%%%%%%%%%%%%%%%%%%%%%%%%%%%%%%%%%%%%%%%%%%%%%%%%%%%%%%%%%%%%

\subsection{Zixun Lu}
I have learned lots of knowledges from this project. In technical knowledge, this project has lots of basic programing
skill train, so I am a better C/C++ programmer. I was learning C/C++ in my first year of university life, and then I
take python and other program languages to finish my work. This project pull me back to the C/C++ programing, and
I never finish this so big project before. It’s struggle at the beginning, but actually I get used to it during I write a code,
and I review my C/C++ programing knowledge before I start write true code. The most important technical knowledge
that I have learned is multiple threads and memory manager. The first year teaching only tech you the basic idea about
the C++, but it is not enough for true program. We usually want have higher speed with our program, so I learned
how to do it. I used the Pthread library which is the POXIS strander to speed up my program. Because we have the
1. We wanted to mount the directory in the VM so we could simply spin up the development VM, write code on our computers with our
favorite IDEs, and just use the VM to build and test without requiring constant re-provisioning or manual copy/pasting of the code into the
VM.
80
in-memory cache for our project, I have learned the C++ memory manager problem. Basically, I am responsible for the
back-end system implementation, and I come to Elijah to integrate back-end to the front-end.
In non-technical knowledge, I have learned how to track the progress of the project, how to manager a project, and
how to communicate with teammate and client. The progress tracking is important, because it 9 month project, it is
impossible we remember everything in our mind. The Capstone project is my first big project in my university life,
so I take time to learn how to manage project. Elijah is kind of experience people, because he was a leader in many
project, I learned lot of things from him. First is the time management for a team, we should have frequency contact
in a team, and then we should come together to talk about the project before we start writing a program. We should
contact the client for clear the requirement early, so it save the time to match our time schdule. The most important
things for a team is that we should keep us in a same page, so none of us confused and did somethings worry. For
the teamwork, I was lucky before. Everyone did the fair job, so we finished our project early. However, this project is
not. I understand that it hard to said I don’t know about this technologies for a student, but I don’t understand why
computer science student dose not learn a technologies. It looks like that he will say that I will finish my work soon, but
actually he never did anything. Thus, we have to re-assign our work, and because our project contain lots of lower-level
programing skills, one of our teammate just did nothing. In this situation, I have leaned if one of our teammate is lack
of technical knowledge, and if we can’t fire it, we should just give him/her a simple job such as writing documentation
or something easy to do.
If I can do my project all over again. I may did the same things as before, but I should have little tips for myself.
• It is better that have a prototype for a big project, because it gives the initial idea how this program works, and then
we improve it to meet our requirement.
• Do not rely on your teammate, even if your teammate is good person, but you should do your work by yourself. If
your teammate is lazy person, and you have to do your work by yourself too.
• I should check the requirement even after I finish my work. I should ask myself that is that good? is that enough?
and then I should talk with client to check if it is good or not.
