% Section Ten: CLI
% Owned by Elijah C. Voigt
\subsubsection{Command Line Interface (Elijah C. Voigt)}

\paragraph{Options}

For the CLI we can develop a native C client, a simple Bash client, or split the difference with a Python client.

\paragraph{Goals for use in design}

Each of these CLI tools must be able to construct an HTTP POST request with two documents, send that request to a server, and parse a response payload.

\paragraph{Criteria being evaluated}

The criteria for this, as it is not part of the core functionality, is simplicity to write, maintain, and use for end-users.

\paragraph{Comparison breakdown}

\begin{table}[H]
  \begin{center}
    \begin{tabular}{ | l | p{10cm} |}
      \hline
      Technology & Description  \\ \hline
      C/C++ \cite{} &
      \begin{itemize}
        \item Keeps the code base exclusively C/C++.
        \item Can be distributed with system package managers.
        \item Difficult to write and maintain.
      \end{itemize}\\ \hline
      Python \cite{} &
      \begin{itemize}
        \item Simple to write and maintain.
        \item Requires few external dependencies.
        \item Can be distributed with the \inlinecode{pip} package manager.
      \end{itemize}\\ \hline
      Bash \cite{} &
      \begin{itemize}
        \item Very simple to write.
        \item Quick to write using existing system tools like \inlinecode{curl} or \inlinecode{wget}.
      \end{itemize}\\ \hline
    \end{tabular}
  \end{center}
  \caption{Technology evaluated for the command-line interface.}
\end{table}

\paragraph{Discussion}

The C/C++ option is not necessary at all.
This is an option which ought to be considered, but ultimately isn't worth pursuing as it will be very complicated to write and maintain, especially when simpler script-based options exist.

Python is a great middle ground.
The language comes with many web-request libraries and provides tools for users to upload their application to the \inlinecode{pip} package manager.
This means we can write a tool which performs well, is easy to maintain, requires few external dependencies, and can be downloaded by a package manager.
It will also be available on all platforms which run python, which includes Unix and Windows systems.

Bash is a viable candidate, especially when other tools like \inlinecode{curl} and \inlinecode{wget} will make quick work of the task.
The downside to this is that we cannot host our CLI on any standard package manager for verifiable distribution.
This forces users to download the CLI from our website directly.
Ultimately this will probably be used as a proof of concept, but it is not a final product.

\paragraph{Selection}

We will start by creating a CLI in Bash for testing purposes, and if time allows we will create a more polished CLI in Python.
Python language offers the best of both worlds in terms of simplicity, maintainability, and ease of use for end users, but for the sake of ``getting something out the door'' we will first create a tool that works in Bash.
