%%%%%%%%%%%%%%%%%%%%%%%%%%%%%%%%%%%%%%%%%%%%%%%%%%%%%%%%%%%%%%%%%%%%%%%%%%%%%%%%
% What did you learn from all this?
% One per team member.
% > What technical information did you learn?
% > What non-technical information did you learn?
% > What have you learned about project work?
% > What have you learned about project management?
% > What have you learned about working in teams?
% > If you could do it all over, what would you do differently?
% Be honest here -- no B.S.
%%%%%%%%%%%%%%%%%%%%%%%%%%%%%%%%%%%%%%%%%%%%%%%%%%%%%%%%%%%%%%%%%%%%%%%%%%%%%%%%

\subsection{Elijah C. Voigt}

I tried early on in this project \textit{not} to be the leader of the group.
I have been the defacto leader of many groups and it is rarely rewarding and is \textit{always} time consuming.
This project was similarly time consuming and not very rewarding, but I did learn a lot of useful lessons along the way.

When someone on your team does not have the technical skills to complete a task, they will not tell you this until you build some level of trust and actually test them.
Once you've built trust with somebody they will tell you ``I do not have the technical skill to do this.''
Before you've built trust there is shame in revealing this, even if you \textit{do not care about their shame}; you just want something to get done as quickly as possible, but they will struggle to get it done until you say ``have you done that thing yet?'' and they will keep saying ``Not yet, but I am closer.''
It's stressful to say you don't know how to do something, going forward I hope to more easily recognize when I am not ready to complete a task knowing now that this is exactly what my manager needs to know.

Once you've established that someone on your team isn't technically proficient enough to complete a task you need to figure out what to do next.
You should (probably) reassign tasks immediately so that everybody is doing what they're capable of.
In addition to this you will probably need to spend one-on-one time with struggling developers to help them get up to speed on their new tasks and to make sure they \textit{actually can} do their new task, otherwise you'll need to re-assign tasks later.

Last I've learned that projects are not designed to be fair.
You're not going to be assigned to a five person project where you do exactly one-fifth of the work.
You might to a little more, or a little less, or even a \textit{lot} more, that's just how it goes.
It doesn't mean that live hates you, it just means you're on a team of people with diverse backgrounds and proficiencies.
You should voice your concerns to your boss if something feels \textit{really} out of whack, but don't take it personally if the person you're working with isn't doing their one-fifth.
Just don't.

I learned a lot of useful technologies, some of which I had a conversational understanding of but had never really dived into myself.
These mostly included Javascript, jQuery, Apache CGI scripting, Linux's systemd, and I got a lot better at writing (read: debugging) C and C++ code.

If I could do this project again I would make the following changes:

\begin{itemize}
  \item \textbf{I would not write skeleton code for the entire project.}
        This was a good idea at the time, but a lot of the code I wrote for this was thrown out because of weird bugs we couldn't solve.
        Instead if we had focused on getting \textit{something out the door} as soon as possible we would have had something to show our client earlier in development and we would have ultimately produced a better product which was more feature rich.
        The product we ended up completing was not functional until every piece had been completed, which took a lot of time and was very difficult to execute on.
        Sometimes you've just got to be a little scrappy to get your project done.
  \item \textbf{I would stick to my lane.}
        I ended up picking up the slack of one of my teammates which caused me a lot of undue stress and wasn't much fun for him either.
        If I had just done what I was responsible for I would have gotten my part of the project done and would have been a little less stressed.
        On the other hand a lot of the project probably would not have gotten done, and the reason I picked up the slack was not because I wanted to do more than my share of work but because I wanted the project to be good.
        Ultimately it's my own damn fault, but it is still something I would try to do differently in future projects.
  \item \textbf{I would have worked to get a more complete requirements list.}
        A lot of the project's requirements either shifted or grew more defined the further into development we went, this caused some headaches.
        For example, parameter and additional file passing was not something we new we needed to add until the end of winter term.
        I asked our client and teammates if we knew that we were meeting our requirement needs, to which everybody said yes, but the question was not quite asked correctly.
        Going back to point one, if we had a demo to show our client we could ask ``Is this good enough?'' and get feedback about additional features we needed instead of blindly walking toward our deadline just to realize (on my own) that we were missing features which were necessary for end users and to benchmark the application.
        This was frustrating more than anything, but honestly if we had just stuck to our original (not really good enough for production) requirements I doubt anybody would care; this is just my personal angst about making something that people will actually use instead of spending a year on a prototype that will be put away in a dusty box.
\end{itemize}
