\subsubsection{Component Overview}
\label{component-overview}

The following is a list of \textbf{required} project components ordered by importance:

\begin{figure}[H]
  \begin{center}
      \begin{tabular}{ | l | r | }
      \hline
        Component & Owner \\ \hline
        Document Transformer & Elijah C. Voigt \\ \hline
        Website Interface & Shuai Peng \\ \hline
        Web API & Elijah C. Voigt \\ \hline
        Document Cache & Shuai Peng \\ \hline
        Daemon Process & Elijah C. Voigt \\ \hline
        Parallel Document Transformation & Shuai Peng\\ \hline
        Benchmarking & Zixun Lu \\ \hline
        Debian Software Package & Zixun Lu \\ \hline
      \end{tabular}
  \end{center}
  \caption{Required project components ordered by importance}
\end{figure}

\begin{figure}[H]
  \begin{center}
      \begin{tabular}{ | l | r | }
      \hline
        Component & Owner \\ \hline
        CLI Interface & Elijah C. Voigt \\ \hline
        CentOS Linux software package & Elijah C. Voigt \\ \hline
        Windows software package & Shuai Peng \\ \hline
        FreeBSD software package & Elijah C. Voigt  \\ \hline
      \end{tabular}
  \end{center}
  \caption{Stretch goal project components ordered by importance}
\end{figure}

%%%%%%%%%%%%%%%%%%%%%%%%%%%%%%%%%%%%%%%%%%%%%%%%%%%%%%%%%%%%%%%%%%%%%%%%%%%%%%%%
%%%%%%%%%%%%%%%%%%%%%%%%%%%%%%%%%%%%%%%%%%%%%%%%%%%%%%%%%%%%%%%%%%%%%%%%%%%%%%%%
%%%%%%%%%%%%%%%%%%%%%%%%%%%%%%%%%%%%%%%%%%%%%%%%%%%%%%%%%%%%%%%%%%%%%%%%%%%%%%%%
\subsubsection{Document Transformer}
\label{document-transformer}

This subsubsection outlines the views and viewpoints related to the Document Transformer.
These components relate to core functionality of the application:

\begin{enumerate}
  \item Taking input \gls{xml} and \gls{xslt} documents
  \item Parsing them into \gls{dom} objects
  \item Transforming those into a new \gls{xml} document
  \item Returning the final product to a user
\end{enumerate}

This subsubsection does not necessarily outline how users interact with the application, for that one should see the User Interface subsubsection (\ref{user-interface}).

%%%%%%%%%%%%%%%%%%%%%%%%%%%%%%%%%%%%%%%%%%%%%%%%%%%%%%%%%%%%%%%%%%%%%%%%%%%%%%%%
%%%%%%%%%%%%%%%%%%%%%%%%%%%%%%%%%%%%%%%%%%%%%%%%%%%%%%%%%%%%%%%%%%%%%%%%%%%%%%%%
\textbf{Transformer}
\label{transformer}

As the name suggests, the Transformer is the core of the XZES40 Document Transformer, which is the core of our application as a whole.
At a high level this component takes two documents, an \gls{xml} and \gls{xslt} document, and returns a transformed \gls{xml} document.
The rest of the application is built around this component; all other parts of the application depend on or work toward this feature.

%%%%%%%%%%%%%%%%%%%%%%%%%%%%%%%%%%%%%%%%%%%%%%%%%%%%%%%%%%%%%%%%%%%%%%%%%%%%%%%%
\textbf{Context}

The Transformer provides the core functionality of the application by transforming input documents into output documents.

All users of the application use the Transformer indirectly by way of using it to transform their input documents into output documents.
The Transformer component is not be directly exposed, however it is accessible via the Web \gls{api} (\ref{web-api}).

%%%%%%%%%%%%%%%%%%%%%%%%%%%%%%%%%%%%%%%%%%%%%%%%%%%%%%%%%%%%%%%%%%%%%%%%%%%%%%%%
\textbf{Composition}

The Transformer's functionality is outlined in the following steps:

\begin{enumerate}
  \item Receive the input \gls{xml} \gls{xslt} files.
  \item Create a Document object out of the input files. This will either encode the input files as DOM objects (InputSource) or fetch the pre-compiled objects from the cache. (\ref{cache} \ref{document})
  \item Pass the Document's parsed contents to the \cite{xalan-library} \textbf{transform} method.
  \item Encode the transformed document back to a text file and return this document.
\end{enumerate}

These steps will each be held in a thread spawned by the long-running daemon process. (\ref{daemon})

%%%%%%%%%%%%%%%%%%%%%%%%%%%%%%%%%%%%%%%%%%%%%%%%%%%%%%%%%%%%%%%%%%%%%%%%%%%%%%%%
\textbf{Dependencies}

The Transformer directly depends on the following internal application components for the following reasons:

\begin{itemize}
  \item {
      The Document class (\ref{parser}) is used to create a transformable \gls{dom} object from an input file. \cite{dom-spec}
  }
  \item {
    The Document Cache (\ref{cache}) is used to store and retrieve parsed and transformed documents.
    While this cache isn't strictly necessary for document transformation, it is used to speed up the process drastically.
  }
\end{itemize}

%%%%%%%%%%%%%%%%%%%%%%%%%%%%%%%%%%%%%%%%%%%%%%%%%%%%%%%%%%%%%%%%%%%%%%%%%%%%%%%%
\textbf{State Dynamics}

The Transformer deals heavily with state dynamics in ways: transforming documents .

\textbf{Transforming documents}

The Transformer does not directly handle transforming documents, this task is delegated to the \gls{xalan} library. \cite{xalan-library}
The transformer takes two or more documents as input: an \gls{xml} \gls{dom} file and an \gls{xslt} object.
The input documents are transformed by the \gls{xalan} library and the resulting document is eventually returned by the writer (\ref{writer}) via the \gls{api} (\ref{web-api}).

%%%%%%%%%%%%%%%%%%%%%%%%%%%%%%%%%%%%%%%%%%%%%%%%%%%%%%%%%%%%%%%%%%%%%%%%%%%%%%%%
\textbf{Interactions}

Few components directly call the Transformer, however the transformer depends heavily on communicating with the cache.
This is done by using a ``cache'' object.
Documents are retrieved by using a ``get'' method and added to the cache with a ``set'' method.
More information can be found in the Cache subsubsection of this document (\ref{cache}).

%%%%%%%%%%%%%%%%%%%%%%%%%%%%%%%%%%%%%%%%%%%%%%%%%%%%%%%%%%%%%%%%%%%%%%%%%%%%%%%%
\textbf{Interfaces}

The Transformer has one programmer-facing interface, the ``transform\_documents( filepaths )'' function.
This can be used for testing, mocking, or implementation purposes.

\textbf{transform\_documents( filepaths )}

\begin{itemize}
  \item This function takes as argument the path to the \gls{xml} file, the path to \gls{xslt} document, the output destination for the output file.
  \item As output it writes a transformed \gls{xml} and returns a status code.
\end{itemize}

%%%%%%%%%%%%%%%%%%%%%%%%%%%%%%%%%%%%%%%%%%%%%%%%%%%%%%%%%%%%%%%%%%%%%%%%%%%%%%%%
%%%%%%%%%%%%%%%%%%%%%%%%%%%%%%%%%%%%%%%%%%%%%%%%%%%%%%%%%%%%%%%%%%%%%%%%%%%%%%%%
\textbf{Document class}
\label{document}
\label{parser}

The Document class gets as input an \gls{xml} or \gls{xslt} file prepares it to be transformed.
It does this by either compiling it into a \gls{dom} objects and adding it to the cache or fetching the already compiled object from the cache.

%%%%%%%%%%%%%%%%%%%%%%%%%%%%%%%%%%%%%%%%%%%%%%%%%%%%%%%%%%%%%%%%%%%%%%%%%%%%%%%%
\textbf{Context}

The Document class is a major component of the application.
It receives as input an \gls{xml} or a \gls{xslt} file, it then parses and store or just retrieves the \gls{dom} representation of that file.
The user will not use this function directly, but it executed by the Transformer (\ref{reader}).

%%%%%%%%%%%%%%%%%%%%%%%%%%%%%%%%%%%%%%%%%%%%%%%%%%%%%%%%%%%%%%%%%%%%%%%%%%%%%%%%
\textbf{Dependencies}

The Document class also depends on the Cache to store and retrieve parsed objects.

%%%%%%%%%%%%%%%%%%%%%%%%%%%%%%%%%%%%%%%%%%%%%%%%%%%%%%%%%%%%%%%%%%%%%%%%%%%%%%%%
\textbf{Interactions}

\begin{enumerate}
  \item The class obtains a \gls{unicode} formatted \gls{xml} or \gls{xslt} file-path.
  \item The class checks if the file has been parsed and stored in the Cache, and does not re-parse the file if it is in the Cache.
  \item The class generates a \gls{dom} object via the \gls{xerces} library.
  \item The class stores the \gls{dom} object to the Cache.
\end{enumerate}

%%%%%%%%%%%%%%%%%%%%%%%%%%%%%%%%%%%%%%%%%%%%%%%%%%%%%%%%%%%%%%%%%%%%%%%%%%%%%%%%
\textbf{Resources}

The \gls{xerces} library will be used during the parsing to generate a \gls{dom} object from the input file.

%%%%%%%%%%%%%%%%%%%%%%%%%%%%%%%%%%%%%%%%%%%%%%%%%%%%%%%%%%%%%%%%%%%%%%%%%%%%%%%%
\textbf{Interfaces}

The following function declaration is used for the parser method:

\textbf{class Document(source\_file\_path)}

This constructor receives an \gls{xml} file or a \gls{xslt} file, and then parses it into a \gls{dom} object which is stored in the Cache.

In parsing and storing the object, the class has methods which read the file contents, hashes the contents of the file, and uses this as a key when inserting data into the Cache.

%%%%%%%%%%%%%%%%%%%%%%%%%%%%%%%%%%%%%%%%%%%%%%%%%%%%%%%%%%%%%%%%%%%%%%%%%%%%%%%%
%%%%%%%%%%%%%%%%%%%%%%%%%%%%%%%%%%%%%%%%%%%%%%%%%%%%%%%%%%%%%%%%%%%%%%%%%%%%%%%%
\textbf{Cache}
\label{cache}

The Cache is a plus one feature of the XZES40 application.
The Cache speeds up document transformation by storing and retrieving previously parsed documents.

In practice the Cache will operate much the same as a \gls{hash-map} does, storing data at a location given a key which can also be used to retrieve the data.
We will using keyList which is provided from XercessC.

%%%%%%%%%%%%%%%%%%%%%%%%%%%%%%%%%%%%%%%%%%%%%%%%%%%%%%%%%%%%%%%%%%%%%%%%%%%%%%%%
\textbf{Context}

The program heavily depends on the Cache.
The Cache can store, delete, and retrieve \gls{dom} object from the in-memory cache 

\textbf{Storing data}

This is the major element of the Cache.

The Cache stores data from the user, primarily parsed \gls{dom} data to avoid re-compiling files.

\textbf{Deleting data}

This provides the ability to remove items from the Cache for whatever reason.

\textbf{Retrieving data}

This allows users to fetch information stored in the Cache given the object's key.

%%%%%%%%%%%%%%%%%%%%%%%%%%%%%%%%%%%%%%%%%%%%%%%%%%%%%%%%%%%%%%%%%%%%%%%%%%%%%%%%
\textbf{Composition}

Below are few of the components making up the Cache.

\begin{itemize}
    \item The Cache stores input data in a \gls{struct} along with the last time it was accessed and they key used to access the data.
    \item The Cache can retrieve the data via searching the associated key.
    \item The Cache can delete data corresponding with a key.
\end{itemize}

%%%%%%%%%%%%%%%%%%%%%%%%%%%%%%%%%%%%%%%%%%%%%%%%%%%%%%%%%%%%%%%%%%%%%%%%%%%%%%%%
\textbf{Logical}

XZES40 handles object as a special structure.
The Cache saves that object in the following format.

\textbf{Key}

This is a hash value associated with a cached object.
This value is for retrieving the data from the Cache.

\textbf{Content}

The content is the parsed or transformed \gls{dom} object.

\begin{lstlisting}[caption={Psuedocode for the caching generic object ``node''.}]
struct node {
  dom data;
  string key;
  date last_used;
}
\end{lstlisting}

%%%%%%%%%%%%%%%%%%%%%%%%%%%%%%%%%%%%%%%%%%%%%%%%%%%%%%%%%%%%%%%%%%%%%%%%%%%%%%%%
\textbf{Information}

If an object is not in the Cache, the Cache stores the object into the in-memory cache. 
Objects are removed from the cache via the Cache class' delete method.

%%%%%%%%%%%%%%%%%%%%%%%%%%%%%%%%%%%%%%%%%%%%%%%%%%%%%%%%%%%%%%%%%%%%%%%%%%%%%%%%
\textbf{State Dynamics}

The Cache handles the following state changes in the following ways.

\begin{itemize}
    \item {
      When the Cache starts it allocates a block of memory for storage.
     }
    \item {
      If an item is being set the cache ignores previously existing data.
      It is the developers duty to ensure important data is not being overwritten.
    }
    \item {
      When data is being read the state of that item in cache is assumed not to change.
      If the item is not found it returns an empty object.
    }
    \item  When an item is deleted it returns a SUCCESS status if the object existed and is not deleted, and a FAILURE status if the object did not exist in the cache before the call.
\end{itemize}

%%%%%%%%%%%%%%%%%%%%%%%%%%%%%%%%%%%%%%%%%%%%%%%%%%%%%%%%%%%%%%%%%%%%%%%%%%%%%%%%
\textbf{Interactions}

The parser sends a parsed \gls{dom} object to the Cache. 
The Cache will check the flies if it is exist in the memory. 
If the file exists in the memory, the Cache retrieves the \gls{dom} object from the Cache and return to the Document class. 
If the file does not exist in the memory, the Document class continues to process the file, and stores parsed file into the Cache.

%%%%%%%%%%%%%%%%%%%%%%%%%%%%%%%%%%%%%%%%%%%%%%%%%%%%%%%%%%%%%%%%%%%%%%%%%%%%%%%%
\textbf{Algorithms}

The Cache can be divided into five major function.
The First function is \textbf{set}, second is \textbf{get}, third is \textbf{dump}, fourth is \textbf{load}, and fifth is \textbf{delete}.

\textbf{Set}

Stores an object at a location in the allocated memory block based on the an \gls{md5} hash of the ``key'' parameter given.

\textbf{Get}

This function retrieves the \gls{dom} object from the cache by the key of the parsed file.

\begin{enumerate}
    \item If there is key exist in the cache, \textbf{return} the \gls{dom} object at that location.
    \item If the key does not exist in the cache, \textbf{return} an empty storage struct object.
\end{enumerate}

\textbf{Delete}

This function delete the \gls{dom} object from the cache by the key of the parsed file.

\begin{enumerate}
    \item The Delete receives a key to delete the data it's located at.
    \item If there is key exist in the cache, the Delete deletes this \gls{dom} object in the cache, and \textbf{return} that there is successfully delete.
    \item If the keys is not key in the cache, \textbf{return} is no \gls{dom} object inside of cache.
\end{enumerate}

%%%%%%%%%%%%%%%%%%%%%%%%%%%%%%%%%%%%%%%%%%%%%%%%%%%%%%%%%%%%%%%%%%%%%%%%%%%%%%%%
\textbf{Resources}

The Cache needs a large memory allocation to function optimally.

%%%%%%%%%%%%%%%%%%%%%%%%%%%%%%%%%%%%%%%%%%%%%%%%%%%%%%%%%%%%%%%%%%%%%%%%%%%%%%%
\textbf{Dependencies}

The Cache depends on the Document and the Transformer to populate it.
The Document class sends a file to the Cache for checking if file exists in the Cache.
The Document class also can send a parsed \gls{dom} object for storing into the Cache if file does not exist in the Cache.
The Cache may either return a \gls{dom} object to the Transformer, or the cache return null if the \gls{dom} object is not in the Cache.

%%%%%%%%%%%%%%%%%%%%%%%%%%%%%%%%%%%%%%%%%%%%%%%%%%%%%%%%%%%%%%%%%%%%%%%%%%%%%%%%
\textbf{Interfaces}

Below are the major interfaces for the cache component of XZES40-Transformer.

\textbf{int document\_cache( dom\_object file )} 

This checks the \gls{dom} object if it exist in the Cache.

\begin{itemize}
    \item If it's not in the cache, return fail. 
    \item If it is in the cache, return true.
\end{itemize}

\textbf{dom\_object set\_cache( dom\_object file )} 

This receives object file and store it into the memory.

\textbf{dom\_object delete\_cache( dom\_object file )}

This deletes data from the Cache.

\begin{itemize}
    \item If the \gls{dom} object exists in the cache, return true.
    \item If the \gls{dom} object does not exist in the Cache, return error.
\end{itemize}

\textbf{dom\_object get\_cache( dom\_object file )} 

This retrieves data from the cache.

\begin{itemize}
    \item If the \gls{dom} object exists in the cache, return the \gls{dom} object.
    \item If the \gls{dom} object does not exist in the Cache, return error.
\end{itemize}

%%%%%%%%%%%%%%%%%%%%%%%%%%%%%%%%%%%%%%%%%%%%%%%%%%%%%%%%%%%%%%%%%%%%%%%%%%%%%%%%
%%%%%%%%%%%%%%%%%%%%%%%%%%%%%%%%%%%%%%%%%%%%%%%%%%%%%%%%%%%%%%%%%%%%%%%%%%%%%%%%
\textbf{Parallel Computation}
\label{parallel-computation}

In addition to the Cache component of XZES40-Transformer (\ref{cache}), the application will also carry out certain computations in parallel to further leverage the computing resources available to it and compile documents even faster.

%%%%%%%%%%%%%%%%%%%%%%%%%%%%%%%%%%%%%%%%%%%%%%%%%%%%%%%%%%%%%%%%%%%%%%%%%%%%%%%%
\textbf{Context}

The Parallel Computation component of the application carries out the following operations in parallel to speed up documentation transformations.

\textbf{Document Parsing}

This will be carried out in parallel.
These operations are logically independent so they can be carried out simultaneously without affecting data integrity.

\textbf{User Requests} 

This will also be carried out in Parallel, handled by Apache.

%%%%%%%%%%%%%%%%%%%%%%%%%%%%%%%%%%%%%%%%%%%%%%%%%%%%%%%%%%%%%%%%%%%%%%%%%%%%%%%%
\textbf{Dependencies}

The Parallel Computation component of the application will be carried out at the high level by Apache delegating parsing jobs, see the \gls{web-api} subsubsection for more information \ref{web-api}.
the application will also carry out parallel computation internally (C++) using the MPI library.

%%%%%%%%%%%%%%%%%%%%%%%%%%%%%%%%%%%%%%%%%%%%%%%%%%%%%%%%%%%%%%%%%%%%%%%%%%%%%%%%
\textbf{Interactions}

Our MPI-based thread computing will operate mostly autonomously, except when putting data into and fetching data from the application Cache (\ref{cache}).
With the exception of interacting with the Cache each parsing thread will not interact with other internal components of the application.

%%%%%%%%%%%%%%%%%%%%%%%%%%%%%%%%%%%%%%%%%%%%%%%%%%%%%%%%%%%%%%%%%%%%%%%%%%%%%%%%
\textbf{Algorithms}

One major concern with handling a cache by multiple threads and processes in parallel is avoiding data corruption.
This is not a problem we have yet solved and further revisions of this document will elaborate on how we will handle this dilemma.

%%%%%%%%%%%%%%%%%%%%%%%%%%%%%%%%%%%%%%%%%%%%%%%%%%%%%%%%%%%%%%%%%%%%%%%%%%%%%%%%
\textbf{Interfaces}

When documents are being parsed each parsed document will spawn it's own thread.
This functionality is in the Document class component of the application in subsubsection \ref{parser}.

%%%%%%%%%%%%%%%%%%%%%%%%%%%%%%%%%%%%%%%%%%%%%%%%%%%%%%%%%%%%%%%%%%%%%%%%%%%%%%%%
%%%%%%%%%%%%%%%%%%%%%%%%%%%%%%%%%%%%%%%%%%%%%%%%%%%%%%%%%%%%%%%%%%%%%%%%%%%%%%%%
\subsubsection{Daemon}
\label{daemon}

%%%%%%%%%%%%%%%%%%%%%%%%%%%%%%%%%%%%%%%%%%%%%%%%%%%%%%%%%%%%%%%%%%%%%%%%%%%%%%%%
\textbf{Context}

Because the application needs to be available continuously, and because it needs to store objects in an in-memory Cache, the application is daemonized.
This means that it continues to run in a suspended state when it is not actively performing document transformations.
When it does receive an incoming request it spawns a thread, does the transformation, and closes the thread.
This allows multiple document transformations to happen in parallel conveniently.

%%%%%%%%%%%%%%%%%%%%%%%%%%%%%%%%%%%%%%%%%%%%%%%%%%%%%%%%%%%%%%%%%%%%%%%%%%%%%%%%
\textbf{Composition}

The daemon is composed of two main components:

\begin{itemize}
  \item {
    A parent thread which will wait for a signal to handle a request.
    This signal is accompanied with some context for what task needs to be performed, i.e., a document transformation.
  }
  \item {
    A child thread is spawned for each incoming signal.
    The child process exits at the end of it's transformation after returning the transformed document or an error which is propagated up through the parent process to the signaling process.
  }
\end{itemize}

%%%%%%%%%%%%%%%%%%%%%%%%%%%%%%%%%%%%%%%%%%%%%%%%%%%%%%%%%%%%%%%%%%%%%%%%%%%%%%%%
\textbf{Interfaces}

The interface to the daemon is the same as that of the daemon.
A CLI on the host running the daemon will accept an XML and XSL document as input, pass these to the daemon, and respond with an error or transformed document.

%%%%%%%%%%%%%%%%%%%%%%%%%%%%%%%%%%%%%%%%%%%%%%%%%%%%%%%%%%%%%%%%%%%%%%%%%%%%%%%%
%%%%%%%%%%%%%%%%%%%%%%%%%%%%%%%%%%%%%%%%%%%%%%%%%%%%%%%%%%%%%%%%%%%%%%%%%%%%%%%%
%%%%%%%%%%%%%%%%%%%%%%%%%%%%%%%%%%%%%%%%%%%%%%%%%%%%%%%%%%%%%%%%%%%%%%%%%%%%%%%%
\subsubsection{User Interface}
\label{user-interface}

The Document Transformer is fine and great, but without a user interface it's not useful.
The following two subsubsections, the \gls{web-api} (\ref{web-api}), Website interface (\ref{website}), and \gls{cli} (\ref{cli}) together outline the ways users will will interact with the system described in the previous subsubsection.

%%%%%%%%%%%%%%%%%%%%%%%%%%%%%%%%%%%%%%%%%%%%%%%%%%%%%%%%%%%%%%%%%%%%%%%%%%%%%%%%
%%%%%%%%%%%%%%%%%%%%%%%%%%%%%%%%%%%%%%%%%%%%%%%%%%%%%%%%%%%%%%%%%%%%%%%%%%%%%%%%
\textbf{\gls{web-api}}
\label{web-api}

The \gls{web-api} is the standardized interface between the user interfaces and an instance of the application running on a host, communicating over \gls{http}.

%%%%%%%%%%%%%%%%%%%%%%%%%%%%%%%%%%%%%%%%%%%%%%%%%%%%%%%%%%%%%%%%%%%%%%%%%%%%%%%%
\textbf{Context}

The users for the \gls{web-api} are those who use the application from the \gls{cli} (\ref{cli}) and (Website \ref{website}).
Both interfaces interact with the XZES40-Transformer host via standard \gls{http} request methods.
But nobody should use the \gls{api} directly as a \gls{ui} is much easier than crafting an \gls{http} POST request.

%%%%%%%%%%%%%%%%%%%%%%%%%%%%%%%%%%%%%%%%%%%%%%%%%%%%%%%%%%%%%%%%%%%%%%%%%%%%%%%%
\textbf{Composition}

The \gls{web-api}is composed of the following components:

\begin{enumerate}
  \item  An \gls{apache} runs on the remote host.\cite{apache-server}
  \item The server manages a \gls{python} \gls{cgi} script which handles accepting requests and sending responses.
  \item The \gls{python} script calls the XZES40-Transformer application locally, passing input documents from a POST request and sending response files via the \gls{cgi} interface.
\end{enumerate}

%%%%%%%%%%%%%%%%%%%%%%%%%%%%%%%%%%%%%%%%%%%%%%%%%%%%%%%%%%%%%%%%%%%%%%%%%%%%%%%%
\textbf{Dependencies}

The \gls{web-api} internally depends on an XZES40-Transformer binary which accepts an input \gls{xml} file, and input \gls{xslt} file, and writes a transformed file to disk in a predictable location.
The binary should also exit with predictable exit codes to communicate any errors or successes.

%%%%%%%%%%%%%%%%%%%%%%%%%%%%%%%%%%%%%%%%%%%%%%%%%%%%%%%%%%%%%%%%%%%%%%%%%%%%%%%%
\textbf{Interfaces}

The \gls{web-api} communicates with remote clients via a standard \gls{http} \gls{api}.
Below are the requests a client can send and the possible responses:

\textbf{POST /api/}

This is a request sent to the \gls{api} endpoint (/api/) with the intention of getting two input files transformed into a new document.
This can respond in the following ways.

\begin{description}
  \item {
      \textbf{200 OK}  Means the transformation was successful.
        This response includes a body containing the transformed file and a \gls{uri} to re-download the response file.
    }
    \item {
      \textbf{400 USER ERROR} Means that the user sent malformed documents.
        This can include a document which does not follow the \gls{xml}/\gls{xslt} standard to a document which does not have a readable character encoding.
        This response includes a body containing an appropriately specific error.
    }
    \item {
      \textbf{500 SERVER ERROR} Means that the server experienced an internal error while processing the request.
         This includes fatal XZES40-Transformer errors.
        This response includes a body containing an appropriately specific error.
    }
\end{description}

The POST request is a request with a Form containing input documents in fields titled \gls{xml} and \gls{xslt} for an \gls{xml} 1.0 formatted document and an \gls{xslt} 1.0 formatted document respectively.

\textbf{GET /api/}

This is a request sent to the \gls{api} endpoint (/api/) with the intention of getting a status of the server. This can respond in the following ways:

\begin{description}
  \item \textbf{200 OK} Means the \gls{web-api} endpoint is active and functioning correctly.
    \item \textbf{404 NOT FOUND} Means that the \gls{web-api} endpoint is not configured correctly or the user is accessing a page which is not available.
    \item \textbf{503 SERVICE UNAVAILABLE} Means that the \gls{web-api} is setup correctly, but the application on the remote host is not operating correctly.
\end{description}

The GET request is an empty GET request to the servers ``/api/'' endpoint.

\textbf{\gls{python} Interfaces}

The following interfaces are used for implementing the above \gls{http} interfaces.

\textbf{process\_request(http-request req)}

This method is used to receive an \gls{http} \gls{api} request.
It does this by reading the request header and deciding based on that to carry out one of the above responses.
In the case of a POST request it uses an ``exec'' call to the local XZES40-Transformer application binary and responds with the error / output file of that application.

%%%%%%%%%%%%%%%%%%%%%%%%%%%%%%%%%%%%%%%%%%%%%%%%%%%%%%%%%%%%%%%%%%%%%%%%%%%%%%%%
\textbf{Resources}

The \gls{web-api} has the following external dependencies:

\begin{itemize}
  \item The \textbf{\gls{apache}} is used to process \gls{http} requests by running the \gls{python} \gls{cgi} script.
    \item \textbf{\gls{python} 2.7+} is the programming language the \gls{web-api} is written in.
    \item \textbf{cgi} is a \gls{python} library for processing web requests in \gls{python}.
    \item \textbf{cgitb} is a \gls{python} library for developing with the cgi library.
    \item \textbf{mod\_wsgi} is an Apache module for interacting with \gls{python}.
    \item \textbf{mod\_python} is another Apache module for interacting with \gls{python}.
\end{itemize}

They should be installed via a system package manager or the \gls{python} package manager (whichever is appropriate) for the \gls{web-api} to operate correctly.

%%%%%%%%%%%%%%%%%%%%%%%%%%%%%%%%%%%%%%%%%%%%%%%%%%%%%%%%%%%%%%%%%%%%%%%%%%%%%%%%
%%%%%%%%%%%%%%%%%%%%%%%%%%%%%%%%%%%%%%%%%%%%%%%%%%%%%%%%%%%%%%%%%%%%%%%%%%%%%%%%
\textbf{Website}
\label{website}

The website interface is the major user interface for the XZES40 application.
Website interface is a \gls{gui} interface for user.

%%%%%%%%%%%%%%%%%%%%%%%%%%%%%%%%%%%%%%%%%%%%%%%%%%%%%%%%%%%%%%%%%%%%%%%%%%%%%%%%
\textbf{Context}

The Website interface is common interface for application.
User can uses website to complete request with few of instructions.
User can upload the \gls{xml}/\gls{xslt} files that they want, and the website will give the feed back.

%%%%%%%%%%%%%%%%%%%%%%%%%%%%%%%%%%%%%%%%%%%%%%%%%%%%%%%%%%%%%%%%%%%%%%%%%%%%%%%%
\textbf{Composition}

The web interface to access the application in a browser.
Here is components of the \gls{web-ui}.

\begin{itemize}
    \item An upload field for the \textbf{\gls{xml}} file.
    \item An upload field for the \textbf{\gls{xslt}} file.
    \item A button for user send the request to server via \textbf{POST} method.
    \item {
       After user sends a request to server the website will respond with a message to user.
        This message may be error or successfully upload or generate download link.
    }
\end{itemize}

%%%%%%%%%%%%%%%%%%%%%%%%%%%%%%%%%%%%%%%%%%%%%%%%%%%%%%%%%%%%%%%%%%%%%%%%%%%%%%%%
\textbf{State Dynamics}

There are few of states for website.

\begin{itemize}
    \item If user upload bad files the website gives feedback that the files were bad.
    \item If user upload files and server transforming the documents the website gives the feedback that user upload files successfully.
    \item If user uploads files and server is down the website gives the warning that transformation service is not working.
    \item If user uploads good files and the server successfully transforms the documents the user is given a download link to the new file.
\end{itemize}

%%%%%%%%%%%%%%%%%%%%%%%%%%%%%%%%%%%%%%%%%%%%%%%%%%%%%%%%%%%%%%%%%%%%%%%%%%%%%%%%
\textbf{Interactions}

The initialization status of website is waiting for user upload files.
After user select flies that they want upload and click the submit button.
Website will send the request to server via the \gls{web-api}.
There are many possible feedback as following.

\begin{itemize}
    \item If the files uploaded are bad the website pops up a warning about the malformed files.
    \item  If the user uploads both an \gls{xml} and \gls{xslt} file the user is alerted that the request is good to go and that the website will generate a download link for them.
    \item If the connection is broken the website will give them a warning that user should check the connection between client and server.
\end{itemize}

%%%%%%%%%%%%%%%%%%%%%%%%%%%%%%%%%%%%%%%%%%%%%%%%%%%%%%%%%%%%%%%%%%%%%%%%%%%%%%%%
\textbf{Interfaces}

The Web interface is XZES40 main interface for user.
The web interface asks user to upload \gls{xml} files and \gls{xslt} files, and gives the feedback of result.
Here is the diagram for website interface.

% \textbf{Web interface Diagram}

%%%%%%%%%%%%%%%%%%%%%%%%%%%%%%%%%%%%%%%%%%%%%%%%%%%%%%%%%%%%%%%%%%%%%%%%%%%%%%%%
%%%%%%%%%%%%%%%%%%%%%%%%%%%%%%%%%%%%%%%%%%%%%%%%%%%%%%%%%%%%%%%%%%%%%%%%%%%%%%%%
\textbf{\gls{cli}}
\label{cli}

The \gls{cli} is used to interact with the system via a text-based terminal/shell interface.

%%%%%%%%%%%%%%%%%%%%%%%%%%%%%%%%%%%%%%%%%%%%%%%%%%%%%%%%%%%%%%%%%%%%%%%%%%%%%%%%
\textbf{Context}

The users of this interface are individuals who either prefer the \gls{cli} over a web interface or for testing purposes as automating tasks with the \gls{cli} is very common.

%%%%%%%%%%%%%%%%%%%%%%%%%%%%%%%%%%%%%%%%%%%%%%%%%%%%%%%%%%%%%%%%%%%%%%%%%%%%%%%%
\textbf{Composition}

The \gls{cli}, written in \gls{python}, and will be composed of the following components.

\begin{itemize}
  \item The main function parses command-line arguments specified below in the ``Interfaces'' part of this subsubsection.
    \item A query is built for the server including the \gls{xml}and \gls{xslt} documents to the specified server.
    \item The query is sent to the server in a POST request.
    \item When the response is received either an error message is displayed to the user or a file is saved locally.
\end{itemize}

%%%%%%%%%%%%%%%%%%%%%%%%%%%%%%%%%%%%%%%%%%%%%%%%%%%%%%%%%%%%%%%%%%%%%%%%%%%%%%%%
\textbf{Resources}

The \gls{cli} depends on the following system dependencies:

\begin{itemize}
  \item {Python 2.7+} is the programming language it will be implemented in, so a Python runtime will be necessary for the application to run.
    \item {Requests} is the standard Python library for interacting with servers over \gls{http}.
    \item {A terminal and \gls{unix} compliant shell} will also be necessary to access the application via the \gls{cli}.
\end{itemize}

%%%%%%%%%%%%%%%%%%%%%%%%%%%%%%%%%%%%%%%%%%%%%%%%%%%%%%%%%%%%%%%%%%%%%%%%%%%%%%%%
\textbf{Interfaces}

The \gls{cli} has the following options available at the command-line:
\begin{figure}[H]
    \begin{lstlisting}
$ xzes40
    --server=<server-url[:port]> # API endpoint and port to be used.
    --xml=<input-file>.xml       # Input XML file
    --xslt=<style-sheet>.xslt    # Input XSLT file
    \end{lstlisting}
    \caption{\gls{cli} Flags. All fields encased in chevrons symbols are required. All fields with square brackets are optional.}
\end{figure}

\textbf{Example Use-cases}

The following are examples of the \gls{cli} in use:

\begin{figure}[H]
    \begin{lstlisting}[caption={}]
$ xzes40 --server=http://xzes40.example.com:8080 \
         --xml=input-file.xml \
         --xslt=style.xslt
Sending input-file.xml and style.xslt to the http://xzes40.example.com:8080
Downloading transformed document to ./xzes40-transformer-2016-11-22.xml
    \end{lstlisting}
    \caption{``Happy path'' use-case without specifying return file name.}
\end{figure}


\begin{figure}[H]
   \begin{lstlisting}
$ xzes40 --server=http://xzes40.example.com:8080 \
         --xml=input-file.xml \
         --xslt=style.xslt \
         --output=transformed-doc.xml
Sending input-file.xml and style.xslt to the http://xzes40.example.com:8080
Downloading transformed document to ./transformed-doc.xml
   \end{lstlisting}
   \caption{``Happy path'' use-case with return file name.}
\end{figure}

\begin{figure}[H]
    \begin{lstlisting}
$ xzes40 --server=http://xzes40.example.com:8080 \
         --xml=input-file.xml \
         --xslt=style.xslt \
Sending input-file.xml and style.xslt to the http://xzes40.example.com:8080
Server responded with error. One of your documents is malformed.
    \end{lstlisting}
    \caption{``Un-happy path'' with malformed document.}
\end{figure}


``Un-happy path'' with bad host.
\begin{figure}[H]
  \begin{lstlisting}
$ xzes40 --server=http://fakesite.com:8080 \
         --xml=input-file.xml --xslt=style.xslt \
Sending input-file.xml and style.xslt to the http://xzes40.example.com:8080
Unable to reach server. Is the host/port correct?
  \end{lstlisting}
  \caption{``Un-happy path'' with bad host.}
\end{figure}

\begin{figure}[H]
  \begin{lstlisting}
$ xzes40  --xml=input-file.xml--xslt=style.xslt
Please specify a host
  \end{lstlisting}
  \caption{``Un-happy path'' with no host.}
\end{figure}

\begin{figure}[H]
  \begin{lstlisting}
$ xzes40  --server=http://xzes40.example.com:8080
Please specify an input xml and/or xslt document.
  \end{lstlisting}
  \caption{``Un-happy path'' with missing input document.}
\end{figure}

%%%%%%%%%%%%%%%%%%%%%%%%%%%%%%%%%%%%%%%%%%%%%%%%%%%%%%%%%%%%%%%%%%%%%%%%%%%%%%%%
%%%%%%%%%%%%%%%%%%%%%%%%%%%%%%%%%%%%%%%%%%%%%%%%%%%%%%%%%%%%%%%%%%%%%%%%%%%%%%%%
%%%%%%%%%%%%%%%%%%%%%%%%%%%%%%%%%%%%%%%%%%%%%%%%%%%%%%%%%%%%%%%%%%%%%%%%%%%%%%%%
\subsubsection{System Requirements}
\label{system-requirements}

The XZES40-Transformer will be required to work on the Debian Linux system.
Once development on Debian is completed the application will be ported to other \gls{os} platforms.
Our team will release \gls{debian}, \gls{centos}, \gls{bsd} and \gls{windows} packages.
  
%%%%%%%%%%%%%%%%%%%%%%%%%%%%%%%%%%%%%%%%%%%%%%%%%%%%%%%%%%%%%%%%%%%%%%%%%%%%%%%%
%%%%%%%%%%%%%%%%%%%%%%%%%%%%%%%%%%%%%%%%%%%%%%%%%%%%%%%%%%%%%%%%%%%%%%%%%%%%%%%%
\textbf{Installation Packages}
\label{installation-packages}

For installation convenience the XZES40 project will provide a \gls{linux}, \gls{bsd}, and \gls{windows} installation packages.

%%%%%%%%%%%%%%%%%%%%%%%%%%%%%%%%%%%%%%%%%%%%%%%%%%%%%%%%%%%%%%%%%%%%%%%%%%%%%%%%
\textbf{Context}

The user can directly download installation packages through the internet and install the XZES40-Transformer on their local systems.
We will create the Debian package initially.
After this, we will create the \gls{centos}, \gls{bsd}, and \gls{windows} packages.
We will upload those packages to the internet and the users can directly download the packages in their operation system.

%%%%%%%%%%%%%%%%%%%%%%%%%%%%%%%%%%%%%%%%%%%%%%%%%%%%%%%%%%%%%%%%%%%%%%%%%%%%%%%%
\textbf{Resources}

The following tools will be used to create the installation packages. 
\begin{figure}[H]
  \begin{center}                                                                       
    \begin{tabular}{ | p{2.5cm} | p{5cm} | p{5cm} | }
    \hline
      Packages & Tools & Description \\ \hline
      \gls{linux} \& \gls{bsd} Packages  & 
      \begin{itemize}
      \item \gls{fpm}
      \end{itemize} &
      \begin{itemize}
      \item Translates packages from one format to another
      \item Allows re-use of other system's packages
      \end{itemize} \\ \hline

      \gls{windows} Packages & 
      \begin{itemize}
      \item \gls{wix}
      \end{itemize} &
      \begin{itemize}
      \item It is a open source project.
      \item It is more stable and security than other tools.
      \item It has steed
      \end{itemize} \\ \hline

    \end{tabular}
\end{center}
\caption{Installation Packages Resources}
\end{figure}
%%%%%%%%%%%%%%%%%%%%%%%%%%%%%%%%%%%%%%%%%%%%%%%%%%%%%%%%%%%%%%%%%%%%%%%%%%%%%%%%
%%%%%%%%%%%%%%%%%%%%%%%%%%%%%%%%%%%%%%%%%%%%%%%%%%%%%%%%%%%%%%%%%%%%%%%%%%%%%%%%
\textbf{\gls{os-api}}
\label{os-agnostic-api}

The \gls{os-api} will be the programs interface with the \gls{os}.
It will be created to make porting the application easier. 

%%%%%%%%%%%%%%%%%%%%%%%%%%%%%%%%%%%%%%%%%%%%%%%%%%%%%%%%%%%%%%%%%%%%%%%%%%%%%%%%
\textbf{Logical}

Any operating system specific operation will be wrapped by the \gls{api}.

%%%%%%%%%%%%%%%%%%%%%%%%%%%%%%%%%%%%%%%%%%%%%%%%%%%%%%%%%%%%%%%%%%%%%%%%%%%%%%%%
\textbf{Interactions}

The application will interact with the host system via an \gls{os-api}.
This means that all operating system specific operations (e.g., read, write, seek, etc) will be done via an \gls{api}.
When the application is compiled on a new target platform (e.g., \gls{linux}, \gls{bsd}, \gls{windows}) a new platform \gls{api} must be created for compatibility.

%%%%%%%%%%%%%%%%%%%%%%%%%%%%%%%%%%%%%%%%%%%%%%%%%%%%%%%%%%%%%%%%%%%%%%%%%%%%%%%%
\textbf{Interfaces}

The \gls{os-api} interface is for the developer.
This interface performs operations by asking the \gls{os-api} to carry out the task and that request is translated to the system-specific system-call. 

%%%%%%%%%%%%%%%%%%%%%%%%%%%%%%%%%%%%%%%%%%%%%%%%%%%%%%%%%%%%%%%%%%%%%%%%%%%%%%%%
%%%%%%%%%%%%%%%%%%%%%%%%%%%%%%%%%%%%%%%%%%%%%%%%%%%%%%%%%%%%%%%%%%%%%%%%%%%%%%%%
\textbf{Performance Benchmark}
\label{performance-benchmark}

%%%%%%%%%%%%%%%%%%%%%%%%%%%%%%%%%%%%%%%%%%%%%%%%%%%%%%%%%%%%%%%%%%%%%%%%%%%%%%%%
\textbf{Context}

We will run test against similar application to determine how fast XZES40 should be.
Throughout development we will put our application through the same paces and compare which is faster.

%%%%%%%%%%%%%%%%%%%%%%%%%%%%%%%%%%%%%%%%%%%%%%%%%%%%%%%%%%%%%%%%%%%%%%%%%%%%%%%%
\textbf{Resources}

We will be bench-marking these applications specifically for the following reasons.

\begin{figure}[H]
  \centering
  \begin{tabular}{ | l | p{10cm} |}
    \hline
    Technology & Description  \\ \hline
    \gls{xalan} \gls{cli}&
    \begin{itemize}
      \item \gls{xalan} uses \gls{xerces} to parse \gls{xml} documents and \gls{xslt}.
      \item The project provides an open source \gls{cli} program to test the project libraries.
      \item Free and Open Source
    \end{itemize} \\ \hline
    Altova &
    \begin{itemize}
      \item To meet industry demands for an ultra-fast processor.
      \item It offers powerful, flexible options for developers including cml, python.
      \item Superior error reporting capabilities include reporting of multiple errors, detailed error descriptions.
    \end{itemize} \\ \hline
  \end{tabular}
  \caption{Bench-marking Resources}
\end{figure}


%%%%%%%%%%%%%%%%%%%%%%%%%%%%%%%%%%%%%%%%%%%%%%%%%%%%%%%%%%%%%%%%%%%%%%%%%%%%%%%%
%%%%%%%%%%%%%%%%%%%%%%%%%%%%%%%%%%%%%%%%%%%%%%%%%%%%%%%%%%%%%%%%%%%%%%%%%%%%%%%%
%%%%%%%%%%%%%%%%%%%%%%%%%%%%%%%%%%%%%%%%%%%%%%%%%%%%%%%%%%%%%%%%%%%%%%%%%%%%%%%%
\subsubsection{Design Rationale}

This subsubsection explains why certain design decisions were made and ``connects the pieces'' of the application.

%%%%%%%%%%%%%%%%%%%%%%%%%%%%%%%%%%%%%%%%%%%%%%%%%%%%%%%%%%%%%%%%%%%%%%%%%%%%%%%%
%%%%%%%%%%%%%%%%%%%%%%%%%%%%%%%%%%%%%%%%%%%%%%%%%%%%%%%%%%%%%%%%%%%%%%%%%%%%%%%%
\textbf{Cache Decisions}

The reason the Cache uses an in-memory system rather than a database or a file-based cache is purely for performance reasons.
Using an in-memory cache over a file-based one should yield faster performance.
As a compromise the cache is dumped to a file periodically to save the state of the cache in case the application daemon (which holds the cache) is restarted.

This may cause problems as the service may run out of memory.
In an attempt to mitigate this a Garbage Collector may be built which is triggered when a certain percentage of the application's allocated memory is used, or on certain time intervals.
 
%%%%%%%%%%%%%%%%%%%%%%%%%%%%%%%%%%%%%%%%%%%%%%%%%%%%%%%%%%%%%%%%%%%%%%%%%%%%%%%%
%%%%%%%%%%%%%%%%%%%%%%%%%%%%%%%%%%%%%%%%%%%%%%%%%%%%%%%%%%%%%%%%%%%%%%%%%%%%%%%%
\textbf{\gls{web-api} Decisions}

The \gls{web-api} connects the application to the outside world in an ideally simple to implement and quick to operate fashion.
The application uses Apache to handle incoming requests, these are passed to a \gls{cgi} script which could be written in anything, we chose \gls{python} because it is easy to write, maintain, and is well supported.

\gls{python} is called by \gls{apache} which then in turn, based on the request, either returns an application status or calls the XZES40 application locally.
In this way \gls{python} with \gls{apache} is a simple, well supported set of tools which expose the application to the outside world.

These tools were not necessarily chosen for their speed, and so a redesign may be necessary if they create a bottle-neck in the request pipeline.

%%%%%%%%%%%%%%%%%%%%%%%%%%%%%%%%%%%%%%%%%%%%%%%%%%%%%%%%%%%%%%%%%%%%%%%%%%%%%%%%
%%%%%%%%%%%%%%%%%%%%%%%%%%%%%%%%%%%%%%%%%%%%%%%%%%%%%%%%%%%%%%%%%%%%%%%%%%%%%%%%
\textbf{Packaging Decisions}

This document outlines the use of \gls{fpm} to create its \gls{linux} and \gls{unix} packages.
This decision was made for convenience and to allow for quick iteration on the package.
In doing research there \textit{are} other tools which can be used to create packages on \gls{centos}, \gls{debian}, and \gls{bsd} but they are very system specific and so would make iterating on the package very difficult, and updates to the software a pain to package.
Using \gls{fpm} we can even automate the build of packages on our \gls{unix}-like systems.

%%%%%%%%%%%%%%%%%%%%%%%%%%%%%%%%%%%%%%%%%%%%%%%%%%%%%%%%%%%%%%%%%%%%%%%%%%%%%%%%
%%%%%%%%%%%%%%%%%%%%%%%%%%%%%%%%%%%%%%%%%%%%%%%%%%%%%%%%%%%%%%%%%%%%%%%%%%%%%%%%
\textbf{Bench-marking decisions}

To demonstrate the competitiveness of the XZES40-Transformer application we chose to compare the performance of this application against the Xalan C++ and the Altova application.

This feels like a fair and balanced comparison as we hope to be competitive with the Altova application but need to ensure we are at least better than the Xalan C++.
