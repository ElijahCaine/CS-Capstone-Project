% Section Two: Document Transformation
% Owned by Shuai Peng
\subsubsection{XML/XSLT Document Transformation}

\paragraph{Option}

There are three options for the XML/XSLT Document Transformation.
The first option is Xalan-C++, and the second option is Saxon C, the third option is Sablotron.

\paragraph{Goals for use in design}

The major function of XZES40-Transformer is XML/XSLT document transforming.
Apache foundation provide many ways to achieve this function.

\paragraph{Criteria being evaluated}

The XML/XSLT document transformation is our major function for XZES40-Transformer application.
We want this transformation compatible with good parse tools, and version of XSLT and XPath.

\paragraph{Comparison breakdown}

The first option is Xalan-C++, this technology is required by our client, and this technology is supported by Apache.
Xalan-C++ is an XSLT processor for transforming XML documents into HTML, text, or other XML document types.
Xalan-C++ contain the XercesC tools, so we don't need consider the parse tools.

The second option is Saxon C, which is Saxon company project.
It performs the same operations as Xalan-C++, but it supports different version of XSLT and Xpath.

The third option technology is Sablotron.
It used same version of XSLT and Xpath with Xalan-C++, but it designed to be as small program.

\begin{center}
    \begin{tabular}{ | l | p{10cm} |}
      \hline
      Technology & Description  \\ \hline

      Xalan-C++ \cite{xalan} &
      \begin{itemize}
        \item Xalan-C++ is open source project developed by Apache. It is implemented by XSLT version 1.0 and XPath version 1.0.
        \item Xalan-C++ uses Xerces-C++ to parse XML documents and XSL style sheets.
      \end{itemize}\\ \hline

      Saxon C \cite{Saxon_c} &
      \begin{itemize}
        \item Saxon C is open source project. It is implemented by XSLT 2.0/3.0 version and XPath version 2.0/3.0.
        \item Saxon C use different parse tools to handle the date.
      \end{itemize}\\ \hline

      Sablotron \cite{Sablotron_intro} &
      \begin{itemize}
        \item Sablotron is open source project by gingerall. It is implemented by XSLT 1.0 and Xpath 1.0.
      \item Sablotron need extra parse tools to complete transformation.
      \end{itemize}\\ \hline
    \end{tabular}
\end{center}

\paragraph{Discussion}

The Xalan-C++ is the most powerful and popular XML/XSLT document transformation library and is used in many propetary as well as open source tools.
It a robust implementation of the W3C recommendations for XSL Transformations (XSLT) and the XML Path Language (XPath).
The Xalan-C++ is continuing update, and it is good tools for transforming documents.
However, Xalan-C++ has poor structure API reference, and it makes developer hard to read and understand.
Thus, developer need spend time on doing research with API.

The Saxon C is a good alternative to Xalan-C++.
Saxon C can handle higher version of XSLT and Xpath, however implements different parse tools, so we need find out other technology.

The Sablotron does the same work as the Xalan-C++, but Sablotron is designed to be as compact and portable as possible.
The size of Sablotron is much small than Xalan-C.
However, Sablotron is old and has not been updated it since 2006.

\paragraph{The Best option}

We will choose Xalan-C++ as our solution technology.
The first reason is that Xalan-C is required by our client.
We want to using Xerces-C as our parse tools, so Xalan-C is the best choice.
Xalan-C++ is easy to use with clear example.

The second reason is that Sablotron is not stable.
Even they are open source project, but the community has forgetten it.
Saxon C is a good second option if client require different tools to complete XML/XSLT document transformation.
