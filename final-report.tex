\documentclass[draftclsnofoot,onecolumn,letterpaper,10pt]{extras/IEEEtran}

%%%%%%%%%%%%%%%%%%%%%%%%%%%%%%%%%%%%%%%%%%%%%%%%%%%%%%%%%%%%%%%%%%%%%%%%%%%%%%%%
% Preamble
%% Note: Files in this directory are included in parent directory docs.
%% This means you need to source local files as `extras/file_name` instead of `file_name`.
%%%%%%%%%%%%%%%%%%%%%%%%%%%%%%%%%%%%%%%%%%%%%%%%%%%%%%%%%%%%%%%%%%%%%%%%%%%%%%%%
\usepackage{graphicx}                                        
\usepackage{amssymb}                                         
\usepackage{amsmath}                                         
\usepackage{amsthm}                                          

\usepackage{alltt}                                           
\usepackage{float}
\usepackage{color}
\usepackage{url}

%% \usepackage{balance}
%% \usepackage[TABBOTCAP, tight]{subfigure}
\usepackage{enumitem}
%% \usepackage{pstricks, pst-node}

\usepackage[margin=0.75in]{geometry}
\geometry{textheight=8.5in, textwidth=6in}

\graphicspath{ {imgs/} } 

\newcommand{\cred}[1]{{\color{red}#1}}
\newcommand{\cblue}[1]{{\color{blue}#1}}
\newcommand{\inlinecode}[1]{\texttt{#1}}

\usepackage{hyperref}
\usepackage{geometry}
\usepackage[english]{babel}
\usepackage{extras/titling}

\newcommand{\subtitle}[1]{%
  \posttitle{%
    \par\end{center}
    \begin{center}\large#1\end{center}
    \vskip0.5em}%
}

\usepackage{listings}
\usepackage{color}

\definecolor{mygreen}{rgb}{0,0.6,0}
\definecolor{mygray}{rgb}{0.5,0.5,0.5}
\definecolor{mymauve}{rgb}{0.58,0,0.82}

\lstset{ %
  backgroundcolor=\color{white},   % choose the background color; you must add \usepackage{color} or \usepackage{xcolor}
  basicstyle=\footnotesize\ttfamily,        % the size of the fonts that are used for the code
  breakatwhitespace=false,         % sets if automatic breaks should only happen at whitespace
  breaklines=true,                 % sets automatic line breaking
  captionpos=b,                    % sets the caption-position to bottom
  commentstyle=\color{mygreen},    % comment style
  deletekeywords={...},            % if you want to delete keywords from the given language
  escapeinside={\%*}{*)},          % if you want to add LaTeX within your code
  extendedchars=true,              % lets you use non-ASCII characters; for 8-bits encodings only, does not work with UTF-8
}


\makeglossaries

\renewcommand{\glossarysection}[2][]{}

\newpsstyle{Important}{fillstyle=solid, fillcolor=red}
\newpsstyle{NotImportant}{fillstyle=vlines }

%%%%%%%%%%%%%%%%%%%%%%%%%%%%%%%%%%%%%%%%%%%%%%%%%%%%%%%%%%%%%%%%%%%%%%%%%%%%%%%%
% You are making this final notebook for several reasons: 
% > This documents what you did and how you did it.
%   This helps us assign grades.
% > It will be used to help others do something similar.
% > It will be useful for future CS Capstone students to see what is expected of them.
% > It will help the OSU CS Department during its next ABET accreditation.
% > We show these to high school students and companies to show them what an education here is like.
%%%%%%%%%%%%%%%%%%%%%%%%%%%%%%%%%%%%%%%%%%%%%%%%%%%%%%%%%%%%%%%%%%%%%%%%%%%%%%%%
% Format
% You are required to generate this document in LaTeX.
% Remember, this is a permanent record of what you did, and should look the part.
% I hope you turn in a document you are proud to have others view.
%%%%%%%%%%%%%%%%%%%%%%%%%%%%%%%%%%%%%%%%%%%%%%%%%%%%%%%%%%%%%%%%%%%%%%%%%%%%%%%%

\title{
	High Performance XML/XSLT Transformation Server \\
    {
    	% Document Sub-title
    	\Large Spring 2017 Final Project Report
    }
}
\author{
  \IEEEauthorblockN{Zixun Lu (luzi),}
  \IEEEauthorblockN{Shuai Peng (pengs),}
  \IEEEauthorblockN{Elijah Voigt (voigte)}
  \IEEEauthorblockA{\\OSU CS Senior Capstone 2016-2017}
}

%%%%%%%%%%%%%%%%%%%%%%%%%%%%%%%%%
% Document
%%%%%%%%%%%%%%%%%%%%%%%%%%%%%%%%%
\begin{document}

%%%%%%%%%%%%%%%%%%%%%%%%%%%%%%%%%
% Title Page
%%%%%%%%%%%%%%%%%%%%%%%%%%%%%%%%%
\maketitle
\begin{abstract}
  The Oregon State University Capstone course, in collaboration with the Apache Software Foundation, worked to create a high performance web-available XML document transformer.
  The end result is a simple XML document transformer accessible through a web-interface, backed by an in-memory caching system and multi-threaded job processor.
\end{abstract}

\clearpage

\tableofcontents

\clearpage

\section{Introduction}

% An introduction to the project.
% > Why was it requested?
% > What is its importance?

% > Who was/were your client(s)?
% > Who requested it?
% > What was the role of the client(s)?
%   (I.e., did they supervise only, or did they participate in doing development) 

The \textit{High Performance XML/XSLT Transformation Server} project was proposed by Steven Hathaway in affiliation with the Apache Software Foundation.
The original task was described as ``Create a high-performance \gls{xml}/\gls{xslt} document transformation server to perform large and repetitive document transformation tasks in a timely manner.
The source libraries will be the Apache Xerces-C and Apache Xalan-C products.
The server should also include support ICU (International Components for Unicode) in \gls{xml} parsers and document serialization.''
The original deliverables were ``[...] a client program that can demonstrate several document transformations using the services of an \gls{xml}/\gls{xslt} transformation server.'' 

An \gls{xml} transformer is an application which takes in two or more structured input files, specifically \gls{xml} files and \gls{xslt} (XML Style Sheets), processes the transformations specified in the \gls{xslt} file, and outputs the resulting \gls{xml} file.
For the purposes of this project it is not entirely important to understand the underlying transformations taking place, since the objective of the project was to implement a web service using existing [Apache maintained] libraries and tools, not to re-implement an \gls{xml} transformer library.

The application's importance comes mainly from large enterprise needs for \gls{xml} document transformation.
Many organizations depend on \gls{xml} data transformations for a variety of tasks, but tend to perform these tasks on local hardware.
By creating an Open Source application which can offload this computation, employees can get more work done (A) not installing an XML transformer application locally and (B) not using their local machine to carry out computation heavy \gls{xml} transformations.

Steven played a role in the development of the application, making himself available for meetings but not actively mentoring the development team. He did explicitly make himself available for mentor ship if members of the development team needed.

\subsection{Development team}
% > Who are the members of your team?
% > What were their roles?

The development team consisted of Zixun Lu, Shuai Peng, and Elijah C. Voigt.
All three students were seniors in the Computer Science program at OSU.
Zixun and Shuai were exchange students from China; Elijah was a local Oregonian.


\section{Glossary}

\printglossary

\newglossaryentry{xml}{
  type=\acronymtype,
  name={XML},
  description={Extensable Markup Language. The human-readable data format used and processed by our application.},
}

\newglossaryentry{xslt}{
  type=\acronymtype,
  name={XSLT},
  description={Extensable Stylesheet Language. The human readable format used to transform documents in our application.},
}

\newglossaryentry{dom}{
  type=\acronymtype,
  name={DOM},
  description={A cross-platform and language-independent application programming interface that treats XML document as a tree structure.},
}

\newglossaryentry{xalan}{
  type=\acronymtype,
  name={Xalan C++},
  description={A library used to XML transformation in C/C++.},
}

\newglossaryentry{xerces}{
  type=\acronymtype,
  name={Xerces C++},
  description={A library used to perform XML transformation in C/C++.},
}

\newglossaryentry{icu}{
  type=\acronymtype,
  name={ICU},
  description={International Components of Unicode. A library process UTF-8 character format document.},
}

\newglossaryentry{api}{
  type=\acronymtype,
  name={API},
  description={Application Programming Interface. Connects the Apache server to upload or return the files.},
}

\newglossaryentry{utf}{
  type=\acronymtype,
  name={UTF-8},
  description={Unicode Text Format. The international standard for encoding test-based data.},
}

\newglossaryentry{unicode} {
  type=\acronymtype,
  name={Unicode},
  description={Provides an unique number for every character.},
}

\newglossaryentry{hash-map}{
  type=\acronymtype,
  name={Hash-map},
  description={Hash map based implementation of the Map interface.},
}

\newglossaryentry{struct}{
  type=\acronymtype,
  name={Struct},
  description={A complex data type declaration.},
}

\newglossaryentry{os-api}{
  type=\acronymtype,
  name={OS-API},
  description={Operating System API. It will be the program's interface with the operating system.},
}

\newglossaryentry{cli}{
  type=\acronymtype,
  name={CLI},
  description={Command Line Interface. A user interface to a computer's operating system.},
}

\newglossaryentry{gui}{
  type=\acronymtype,
  name={GUI},
  description={Graphical Interface. A user interface to a computer application.},
}

\newglossaryentry{web-ui}{
  type=\acronymtype,
  name={Web-UI},
  description={Web UI. A user interface which uses a browser to render its content.},
}
\newglossaryentry{http}{
  type=\acronymtype,
  name={HTTP},
  description={Hypertext Transfer Protocol. XZES40-Transformer interact with remote clients.},
}

\newglossaryentry{web-api}{
  type=\acronymtype,
  name={Web-API},
  description={Web API. Connects web application by Apache server to upload or return the files.},
}

\newglossaryentry{md5}{
  type=\acronymtype,
  name={MD5},
  description={A complex hashing algorithm used to securely verify data is consistent.},
}

\newglossaryentry{unix}{
  type=\acronymtype,
  name={UNIX},
  description={A family of Operating Systems encompassing Linux, BSD, and MacOS.},
}

\newglossaryentry{ui}{
  type=\acronymtype,
  name={UI},
  description={User Interface. Exposes some functionality, usually referring to one on a computer, to the user through some way of interaction either mouse, keyboard, or combination of the two.},
}

\newglossaryentry{apache}{
  type=\acronymtype,
  name={Apache},
  description={An application for sending and receiving HTTP requests on a remote host.},
}

\newglossaryentry{python}{
  type=\acronymtype,
  name={Python},
  description={A program scripting language used for a wide variety of purposes from scientific applications to dynamic websites.},
}

\newglossaryentry{cgi}{
  type=\acronymtype,
  name={CGI},
  description={Common Gateway Interface. A way for a script on a local host to be run remotely via a web-server like Apache.},
}

\newglossaryentry{uri}{
  type=\acronymtype,
  name={URI},
  description={Universal Resource Indicator. Also called a url, this is the path a client needs to use to identify a web-api endpoint.},
}

\newglossaryentry{os}{
  type=\acronymtype,
  name={OS},
  description={Operating System. Software which runs other software.},
}

\newglossaryentry{debian}{
  type=\acronymtype,
  name={Debian},
  description={A popular Linux operating system.},
}

\newglossaryentry{centos}{
  type=\acronymtype,
  name={CentOS},
  description={A popular Linux operating system based on Red Hat Enterprise Linux.},
}

\newglossaryentry{windows}{
  type=\acronymtype,
  name={Windows},
  description={A very popular non-unix operating system.},
}

\newglossaryentry{bsd}{
  type=\acronymtype,
  name={BSD},
  description={Berkeley Software Distribution. A popular UNIX operating system.},
}

\newglossaryentry{linux}{
  type=\acronymtype,
  name={Linux},
  description={A popular UNIX-like operating system.},
}

\newglossaryentry{fpm}{
  type=\acronymtype,
  name={FPM},
  description={Effing Package Management. A software package creator which targets multiple Unix-like operating systems package managers.},
}

\newglossaryentry{wix}{
  type=\acronymtype,
  name={WIX},
  description={Windows Installer XML. A Windows software packaging program.},
}

\newglossaryentry{foss}{
  type=\acronymtype,
  name={FOSS},
  description={Free and Open Source Software. Software which is developed and maintained for free and by a community.},
}


\section{References}

\printbibliography


\section{Project requirements}
%%%%%%%%%%%%%%%%%%%%%%%%%%%%%%%%%%%%%%%%%%%%%%%%%%%%%%%%%%%%%%%%%%%%%%%%%%%%%%%%
% Introduction.
% What the project reqiurements are.
%%%%%%%%%%%%%%%%%%%%%%%%%%%%%%%%%%%%%%%%%%%%%%%%%%%%%%%%%%%%%%%%%%%%%%%%%%%%%%%%

\subsection{Original project requirements}
%%%%%%%%%%%%%%%%%%%%%%%%%%%%%%%%%%%%%%%%%%%%%%%%%%%%%%%%%%%%%%%%%%%%%%%%%%%%%%%%
% Your original Requirements Document.

% > This needs to be the original document, showing what you thought, at the time, was the project definition.
% > This needs to include the original Gantt chart.
%%%%%%%%%%%%%%%%%%%%%%%%%%%%%%%%%%%%%%%%%%%%%%%%%%%%%%%%%%%%%%%%%%%%%%%%%%%%%%%%

\subsection{Updated project requirements}
%%%%%%%%%%%%%%%%%%%%%%%%%%%%%%%%%%%%%%%%%%%%%%%%%%%%%%%%%%%%%%%%%%%%%%%%%%%%%%%%
% How did the project change since the original Client Requirements Document?
%%%%%%%%%%%%%%%%%%%%%%%%%%%%%%%%%%%%%%%%%%%%%%%%%%%%%%%%%%%%%%%%%%%%%%%%%%%%%%%%

\paragraph{Added requirements}
%%%%%%%%%%%%%%%%%%%%%%%%%%%%%%%%%%%%%%%%%%%%%%%%%%%%%%%%%%%%%%%%%%%%%%%%%%%%%%%%
% What new requirements were added?
% Why?
% Use the following table format:
% | 1 | Requirement | What happened to it | Comments
% | 2 | Requirement | What happened to it | Comments
% | 3 | Requirement | What happened to it | Comments
%%%%%%%%%%%%%%%%%%%%%%%%%%%%%%%%%%%%%%%%%%%%%%%%%%%%%%%%%%%%%%%%%%%%%%%%%%%%%%%%

\paragraph{Updated requirements}
%%%%%%%%%%%%%%%%%%%%%%%%%%%%%%%%%%%%%%%%%%%%%%%%%%%%%%%%%%%%%%%%%%%%%%%%%%%%%%%%
% What existing requirements were changed?
% Why?
% Use the following table format:
% | 1 | Requirement | What happened to it | Comments
% | 2 | Requirement | What happened to it | Comments
% | 3 | Requirement | What happened to it | Comments
%%%%%%%%%%%%%%%%%%%%%%%%%%%%%%%%%%%%%%%%%%%%%%%%%%%%%%%%%%%%%%%%%%%%%%%%%%%%%%%%

\paragraph{Removed requirements}
%%%%%%%%%%%%%%%%%%%%%%%%%%%%%%%%%%%%%%%%%%%%%%%%%%%%%%%%%%%%%%%%%%%%%%%%%%%%%%%%
% What existing requirements were deleted?
% Why?
% Use the following table format:
% | 1 | Requirement | What happened to it | Comments
% | 2 | Requirement | What happened to it | Comments
% | 3 | Requirement | What happened to it | Comments
%%%%%%%%%%%%%%%%%%%%%%%%%%%%%%%%%%%%%%%%%%%%%%%%%%%%%%%%%%%%%%%%%%%%%%%%%%%%%%%%


\section{Project design}
%%%%%%%%%%%%%%%%%%%%%%%%%%%%%%%%%%%%%%%%%%%%%%%%%%%%%%%%%%%%%%%%%%%%%%%%%%%%%%%%
% Introduction.
% What the project design is.
%%%%%%%%%%%%%%%%%%%%%%%%%%%%%%%%%%%%%%%%%%%%%%%%%%%%%%%%%%%%%%%%%%%%%%%%%%%%%%%%

\subsection{Preliminary project design}
%%%%%%%%%%%%%%%%%%%%%%%%%%%%%%%%%%%%%%%%%%%%%%%%%%%%%%%%%%%%%%%%%%%%%%%%%%%%%%%%
% Your original design document
%%%%%%%%%%%%%%%%%%%%%%%%%%%%%%%%%%%%%%%%%%%%%%%%%%%%%%%%%%%%%%%%%%%%%%%%%%%%%%%%
\subsubsection{Introduction}

\paragraph{Purpose}

The purpose of this document is to outline the entirety of the design of the XZES40 application for the purposes of referencing during development and for communication with the project sponsor(s).

\paragraph{Scope}

The scope of this document is to outline in necessarily complex terms how the XZES40-Transformer application will be developed.

\paragraph{Development Time-line}

The following Gantt chart outlines the projected time-line for development.
This starts at the first week of January (winter-term week 1), and goes through project completion in June (end of spring term).

\begin{figure}
    \begin{PstGanttChart}[yunit=1,
                          xunit=1,
                          ChartUnitIntervalName=W,
                          ChartUnitBasicIntervalName=W,
                          TaskUnitIntervalValue=10,
                          TaskUnitType=W,
                          ChartStartInterval=1,
                          ChartShowIntervals]{20}{19}
        \psset{gradangle=180}
        \PstGanttTask[TaskStyle=Important,TaskInsideLabel={Alpha}]{0}{6}
        \PstGanttTask[TaskInsideLabel={Benchmark Data Collection}]{0}{3}
        \PstGanttTask[TaskInsideLabel={Core Functionality}]{0}{11}
        \PstGanttTask[TaskInsideLabel={Basic Transformation}]{0}{3}
        \PstGanttTask[TaskInsideLabel={Caching}]{3}{7}
        \PstGanttTask[TaskInsideLabel={Parallel Computation}]{3}{7}
        \PstGanttTask[TaskInsideLabel={Demo}]{5}{1}
        \PstGanttTask[TaskStyle=Important,TaskInsideLabel={Beta}]{6}{5}
        \PstGanttTask[TaskInsideLabel={CGI Script}]{7}{1}
        \PstGanttTask[TaskInsideLabel={Web Interface}]{8}{2}
        \PstGanttTask[TaskInsideLabel={Debian Package}]{9}{2}
        \PstGanttTask[TaskInsideLabel={Demo}]{10}{1}
        \PstGanttTask[TaskStyle=Important,TaskInsideLabel={Release}]{12}{7}
        \PstGanttTask[TaskInsideLabel={CLI Interface}]{12}{2}
        \PstGanttTask[TaskInsideLabel={RedHat Package}]{12}{6}
        \PstGanttTask[TaskInsideLabel={BSD Package}]{12}{6}
        \PstGanttTask[TaskInsideLabel={Windows Package}]{12}{6}
        \PstGanttTask[TaskInsideLabel={Demo}]{17}{2}
        \PstGanttTask[TaskStyle=NotImportant]{11}{1}
    \end{PstGanttChart}
    \caption{Development Gantt Chart Timeline.}
\end{figure}

\paragraph{Summary}

XZES40-Transformer is an application which transforms one XML with one XSLT document for the purposes of data transformation, like that of a spreadsheet program like Excel.
The uses for this rage widely from business to scientific to personal uses.
For most needs, an XML/XSLT document transformer needs to be able to handle a high volume of requests by a wide variety of clients.
To address these needs the XZES40 team is building an Open Source XML/XSLT document transformer with key optimizations built in to improve the document transformation process; helping businesses, institutions, and individuals get more done in a day.
\cite{xml-spec} \cite{xslt-spec} 

\paragraph{Issuing organization}

This document has been issued by Oregon State University and the Apache Software Foundation through the OSU 2016-2017 CS Capstone class.

\paragraph{Change history}
\begin{figure}
  \begin{tabular}{ | p{5cm} | p{5cm} |}
  \hline
    Revision & Date\\ \hline
    Working Draft & 2016-11-17 \\ \hline
  \end{tabular}
  \caption{Change History Table}
\end{figure}


\subsubsection{Component Overview}
\label{component-overview}

The following is a list of \textbf{required} project components ordered by importance:

\begin{figure}[H]
  \begin{center}
      \begin{tabular}{ | l | r | }
      \hline
        Component & Owner \\ \hline
        Document Transformer & Elijah C. Voigt \\ \hline
        Website Interface & Shuai Peng \\ \hline
        Web API & Elijah C. Voigt \\ \hline
        Document Cache & Shuai Peng \\ \hline
        Daemon Process & Elijah C. Voigt \\ \hline
        Parallel Document Transformation & Shuai Peng\\ \hline
        Benchmarking & Zixun Lu \\ \hline
        Debian Software Package & Zixun Lu \\ \hline
      \end{tabular}
  \end{center}
  \caption{Required project components ordered by importance}
\end{figure}

\begin{figure}[H]
  \begin{center}
      \begin{tabular}{ | l | r | }
      \hline
        Component & Owner \\ \hline
        CLI Interface & Elijah C. Voigt \\ \hline
        CentOS Linux software package & Elijah C. Voigt \\ \hline
        Windows software package & Shuai Peng \\ \hline
        FreeBSD software package & Elijah C. Voigt  \\ \hline
      \end{tabular}
  \end{center}
  \caption{Stretch goal project components ordered by importance}
\end{figure}

%%%%%%%%%%%%%%%%%%%%%%%%%%%%%%%%%%%%%%%%%%%%%%%%%%%%%%%%%%%%%%%%%%%%%%%%%%%%%%%%
%%%%%%%%%%%%%%%%%%%%%%%%%%%%%%%%%%%%%%%%%%%%%%%%%%%%%%%%%%%%%%%%%%%%%%%%%%%%%%%%
%%%%%%%%%%%%%%%%%%%%%%%%%%%%%%%%%%%%%%%%%%%%%%%%%%%%%%%%%%%%%%%%%%%%%%%%%%%%%%%%
\subsubsection{Document Transformer}
\label{document-transformer}

This subsubsection outlines the views and viewpoints related to the Document Transformer.
These components relate to core functionality of the application:

\begin{enumerate}
  \item Taking input \gls{xml} and \gls{xslt} documents
  \item Parsing them into \gls{dom} objects
  \item Transforming those into a new \gls{xml} document
  \item Returning the final product to a user
\end{enumerate}

This subsubsection does not necessarily outline how users interact with the application, for that one should see the User Interface subsubsection (\ref{user-interface}).

%%%%%%%%%%%%%%%%%%%%%%%%%%%%%%%%%%%%%%%%%%%%%%%%%%%%%%%%%%%%%%%%%%%%%%%%%%%%%%%%
%%%%%%%%%%%%%%%%%%%%%%%%%%%%%%%%%%%%%%%%%%%%%%%%%%%%%%%%%%%%%%%%%%%%%%%%%%%%%%%%
\textbf{Transformer}
\label{transformer}

As the name suggests, the Transformer is the core of the XZES40 Document Transformer, which is the core of our application as a whole.
At a high level this component takes two documents, an \gls{xml} and \gls{xslt} document, and returns a transformed \gls{xml} document.
The rest of the application is built around this component; all other parts of the application depend on or work toward this feature.

%%%%%%%%%%%%%%%%%%%%%%%%%%%%%%%%%%%%%%%%%%%%%%%%%%%%%%%%%%%%%%%%%%%%%%%%%%%%%%%%
\textbf{Context}

The Transformer provides the core functionality of the application by transforming input documents into output documents.

All users of the application use the Transformer indirectly by way of using it to transform their input documents into output documents.
The Transformer component is not be directly exposed, however it is accessible via the Web \gls{api} (\ref{web-api}).

%%%%%%%%%%%%%%%%%%%%%%%%%%%%%%%%%%%%%%%%%%%%%%%%%%%%%%%%%%%%%%%%%%%%%%%%%%%%%%%%
\textbf{Composition}

The Transformer's functionality is outlined in the following steps:

\begin{enumerate}
  \item Receive the input \gls{xml} \gls{xslt} files.
  \item Create a Document object out of the input files. This will either encode the input files as DOM objects (InputSource) or fetch the pre-compiled objects from the cache. (\ref{cache} \ref{document})
  \item Pass the Document's parsed contents to the \cite{xalan-library} \textbf{transform} method.
  \item Encode the transformed document back to a text file and return this document.
\end{enumerate}

These steps will each be held in a thread spawned by the long-running daemon process. (\ref{daemon})

%%%%%%%%%%%%%%%%%%%%%%%%%%%%%%%%%%%%%%%%%%%%%%%%%%%%%%%%%%%%%%%%%%%%%%%%%%%%%%%%
\textbf{Dependencies}

The Transformer directly depends on the following internal application components for the following reasons:

\begin{itemize}
  \item {
      The Document class (\ref{parser}) is used to create a transformable \gls{dom} object from an input file. \cite{dom-spec}
  }
  \item {
    The Document Cache (\ref{cache}) is used to store and retrieve parsed and transformed documents.
    While this cache isn't strictly necessary for document transformation, it is used to speed up the process drastically.
  }
\end{itemize}

%%%%%%%%%%%%%%%%%%%%%%%%%%%%%%%%%%%%%%%%%%%%%%%%%%%%%%%%%%%%%%%%%%%%%%%%%%%%%%%%
\textbf{State Dynamics}

The Transformer deals heavily with state dynamics in ways: transforming documents .

\textbf{Transforming documents}

The Transformer does not directly handle transforming documents, this task is delegated to the \gls{xalan} library. \cite{xalan-library}
The transformer takes two or more documents as input: an \gls{xml} \gls{dom} file and an \gls{xslt} object.
The input documents are transformed by the \gls{xalan} library and the resulting document is eventually returned by the writer (\ref{writer}) via the \gls{api} (\ref{web-api}).

%%%%%%%%%%%%%%%%%%%%%%%%%%%%%%%%%%%%%%%%%%%%%%%%%%%%%%%%%%%%%%%%%%%%%%%%%%%%%%%%
\textbf{Interactions}

Few components directly call the Transformer, however the transformer depends heavily on communicating with the cache.
This is done by using a ``cache'' object.
Documents are retrieved by using a ``get'' method and added to the cache with a ``set'' method.
More information can be found in the Cache subsubsection of this document (\ref{cache}).

%%%%%%%%%%%%%%%%%%%%%%%%%%%%%%%%%%%%%%%%%%%%%%%%%%%%%%%%%%%%%%%%%%%%%%%%%%%%%%%%
\textbf{Interfaces}

The Transformer has one programmer-facing interface, the ``transform\_documents( filepaths )'' function.
This can be used for testing, mocking, or implementation purposes.

\textbf{transform\_documents( filepaths )}

\begin{itemize}
  \item This function takes as argument the path to the \gls{xml} file, the path to \gls{xslt} document, the output destination for the output file.
  \item As output it writes a transformed \gls{xml} and returns a status code.
\end{itemize}

%%%%%%%%%%%%%%%%%%%%%%%%%%%%%%%%%%%%%%%%%%%%%%%%%%%%%%%%%%%%%%%%%%%%%%%%%%%%%%%%
%%%%%%%%%%%%%%%%%%%%%%%%%%%%%%%%%%%%%%%%%%%%%%%%%%%%%%%%%%%%%%%%%%%%%%%%%%%%%%%%
\textbf{Document class}
\label{document}
\label{parser}

The Document class gets as input an \gls{xml} or \gls{xslt} file prepares it to be transformed.
It does this by either compiling it into a \gls{dom} objects and adding it to the cache or fetching the already compiled object from the cache.

%%%%%%%%%%%%%%%%%%%%%%%%%%%%%%%%%%%%%%%%%%%%%%%%%%%%%%%%%%%%%%%%%%%%%%%%%%%%%%%%
\textbf{Context}

The Document class is a major component of the application.
It receives as input an \gls{xml} or a \gls{xslt} file, it then parses and store or just retrieves the \gls{dom} representation of that file.
The user will not use this function directly, but it executed by the Transformer (\ref{reader}).

%%%%%%%%%%%%%%%%%%%%%%%%%%%%%%%%%%%%%%%%%%%%%%%%%%%%%%%%%%%%%%%%%%%%%%%%%%%%%%%%
\textbf{Dependencies}

The Document class also depends on the Cache to store and retrieve parsed objects.

%%%%%%%%%%%%%%%%%%%%%%%%%%%%%%%%%%%%%%%%%%%%%%%%%%%%%%%%%%%%%%%%%%%%%%%%%%%%%%%%
\textbf{Interactions}

\begin{enumerate}
  \item The class obtains a \gls{unicode} formatted \gls{xml} or \gls{xslt} file-path.
  \item The class checks if the file has been parsed and stored in the Cache, and does not re-parse the file if it is in the Cache.
  \item The class generates a \gls{dom} object via the \gls{xerces} library.
  \item The class stores the \gls{dom} object to the Cache.
\end{enumerate}

%%%%%%%%%%%%%%%%%%%%%%%%%%%%%%%%%%%%%%%%%%%%%%%%%%%%%%%%%%%%%%%%%%%%%%%%%%%%%%%%
\textbf{Resources}

The \gls{xerces} library will be used during the parsing to generate a \gls{dom} object from the input file.

%%%%%%%%%%%%%%%%%%%%%%%%%%%%%%%%%%%%%%%%%%%%%%%%%%%%%%%%%%%%%%%%%%%%%%%%%%%%%%%%
\textbf{Interfaces}

The following function declaration is used for the parser method:

\textbf{class Document(source\_file\_path)}

This constructor receives an \gls{xml} file or a \gls{xslt} file, and then parses it into a \gls{dom} object which is stored in the Cache.

In parsing and storing the object, the class has methods which read the file contents, hashes the contents of the file, and uses this as a key when inserting data into the Cache.

%%%%%%%%%%%%%%%%%%%%%%%%%%%%%%%%%%%%%%%%%%%%%%%%%%%%%%%%%%%%%%%%%%%%%%%%%%%%%%%%
%%%%%%%%%%%%%%%%%%%%%%%%%%%%%%%%%%%%%%%%%%%%%%%%%%%%%%%%%%%%%%%%%%%%%%%%%%%%%%%%
\textbf{Cache}
\label{cache}

The Cache is a plus one feature of the XZES40 application.
The Cache speeds up document transformation by storing and retrieving previously parsed documents.

In practice the Cache will operate much the same as a \gls{hash-map} does, storing data at a location given a key which can also be used to retrieve the data.
We will using keyList which is provided from XercessC.

%%%%%%%%%%%%%%%%%%%%%%%%%%%%%%%%%%%%%%%%%%%%%%%%%%%%%%%%%%%%%%%%%%%%%%%%%%%%%%%%
\textbf{Context}

The program heavily depends on the Cache.
The Cache can store, delete, and retrieve \gls{dom} object from the in-memory cache 

\textbf{Storing data}

This is the major element of the Cache.

The Cache stores data from the user, primarily parsed \gls{dom} data to avoid re-compiling files.

\textbf{Deleting data}

This provides the ability to remove items from the Cache for whatever reason.

\textbf{Retrieving data}

This allows users to fetch information stored in the Cache given the object's key.

%%%%%%%%%%%%%%%%%%%%%%%%%%%%%%%%%%%%%%%%%%%%%%%%%%%%%%%%%%%%%%%%%%%%%%%%%%%%%%%%
\textbf{Composition}

Below are few of the components making up the Cache.

\begin{itemize}
    \item The Cache stores input data in a \gls{struct} along with the last time it was accessed and they key used to access the data.
    \item The Cache can retrieve the data via searching the associated key.
    \item The Cache can delete data corresponding with a key.
\end{itemize}

%%%%%%%%%%%%%%%%%%%%%%%%%%%%%%%%%%%%%%%%%%%%%%%%%%%%%%%%%%%%%%%%%%%%%%%%%%%%%%%%
\textbf{Logical}

XZES40 handles object as a special structure.
The Cache saves that object in the following format.

\textbf{Key}

This is a hash value associated with a cached object.
This value is for retrieving the data from the Cache.

\textbf{Content}

The content is the parsed or transformed \gls{dom} object.

\begin{lstlisting}[caption={Psuedocode for the caching generic object ``node''.}]
struct node {
  dom data;
  string key;
  date last_used;
}
\end{lstlisting}

%%%%%%%%%%%%%%%%%%%%%%%%%%%%%%%%%%%%%%%%%%%%%%%%%%%%%%%%%%%%%%%%%%%%%%%%%%%%%%%%
\textbf{Information}

If an object is not in the Cache, the Cache stores the object into the in-memory cache. 
Objects are removed from the cache via the Cache class' delete method.

%%%%%%%%%%%%%%%%%%%%%%%%%%%%%%%%%%%%%%%%%%%%%%%%%%%%%%%%%%%%%%%%%%%%%%%%%%%%%%%%
\textbf{State Dynamics}

The Cache handles the following state changes in the following ways.

\begin{itemize}
    \item {
      When the Cache starts it allocates a block of memory for storage.
     }
    \item {
      If an item is being set the cache ignores previously existing data.
      It is the developers duty to ensure important data is not being overwritten.
    }
    \item {
      When data is being read the state of that item in cache is assumed not to change.
      If the item is not found it returns an empty object.
    }
    \item  When an item is deleted it returns a SUCCESS status if the object existed and is not deleted, and a FAILURE status if the object did not exist in the cache before the call.
\end{itemize}

%%%%%%%%%%%%%%%%%%%%%%%%%%%%%%%%%%%%%%%%%%%%%%%%%%%%%%%%%%%%%%%%%%%%%%%%%%%%%%%%
\textbf{Interactions}

The parser sends a parsed \gls{dom} object to the Cache. 
The Cache will check the flies if it is exist in the memory. 
If the file exists in the memory, the Cache retrieves the \gls{dom} object from the Cache and return to the Document class. 
If the file does not exist in the memory, the Document class continues to process the file, and stores parsed file into the Cache.

%%%%%%%%%%%%%%%%%%%%%%%%%%%%%%%%%%%%%%%%%%%%%%%%%%%%%%%%%%%%%%%%%%%%%%%%%%%%%%%%
\textbf{Algorithms}

The Cache can be divided into five major function.
The First function is \textbf{set}, second is \textbf{get}, third is \textbf{dump}, fourth is \textbf{load}, and fifth is \textbf{delete}.

\textbf{Set}

Stores an object at a location in the allocated memory block based on the an \gls{md5} hash of the ``key'' parameter given.

\textbf{Get}

This function retrieves the \gls{dom} object from the cache by the key of the parsed file.

\begin{enumerate}
    \item If there is key exist in the cache, \textbf{return} the \gls{dom} object at that location.
    \item If the key does not exist in the cache, \textbf{return} an empty storage struct object.
\end{enumerate}

\textbf{Delete}

This function delete the \gls{dom} object from the cache by the key of the parsed file.

\begin{enumerate}
    \item The Delete receives a key to delete the data it's located at.
    \item If there is key exist in the cache, the Delete deletes this \gls{dom} object in the cache, and \textbf{return} that there is successfully delete.
    \item If the keys is not key in the cache, \textbf{return} is no \gls{dom} object inside of cache.
\end{enumerate}

%%%%%%%%%%%%%%%%%%%%%%%%%%%%%%%%%%%%%%%%%%%%%%%%%%%%%%%%%%%%%%%%%%%%%%%%%%%%%%%%
\textbf{Resources}

The Cache needs a large memory allocation to function optimally.

%%%%%%%%%%%%%%%%%%%%%%%%%%%%%%%%%%%%%%%%%%%%%%%%%%%%%%%%%%%%%%%%%%%%%%%%%%%%%%%
\textbf{Dependencies}

The Cache depends on the Document and the Transformer to populate it.
The Document class sends a file to the Cache for checking if file exists in the Cache.
The Document class also can send a parsed \gls{dom} object for storing into the Cache if file does not exist in the Cache.
The Cache may either return a \gls{dom} object to the Transformer, or the cache return null if the \gls{dom} object is not in the Cache.

%%%%%%%%%%%%%%%%%%%%%%%%%%%%%%%%%%%%%%%%%%%%%%%%%%%%%%%%%%%%%%%%%%%%%%%%%%%%%%%%
\textbf{Interfaces}

Below are the major interfaces for the cache component of XZES40-Transformer.

\textbf{int document\_cache( dom\_object file )} 

This checks the \gls{dom} object if it exist in the Cache.

\begin{itemize}
    \item If it's not in the cache, return fail. 
    \item If it is in the cache, return true.
\end{itemize}

\textbf{dom\_object set\_cache( dom\_object file )} 

This receives object file and store it into the memory.

\textbf{dom\_object delete\_cache( dom\_object file )}

This deletes data from the Cache.

\begin{itemize}
    \item If the \gls{dom} object exists in the cache, return true.
    \item If the \gls{dom} object does not exist in the Cache, return error.
\end{itemize}

\textbf{dom\_object get\_cache( dom\_object file )} 

This retrieves data from the cache.

\begin{itemize}
    \item If the \gls{dom} object exists in the cache, return the \gls{dom} object.
    \item If the \gls{dom} object does not exist in the Cache, return error.
\end{itemize}

%%%%%%%%%%%%%%%%%%%%%%%%%%%%%%%%%%%%%%%%%%%%%%%%%%%%%%%%%%%%%%%%%%%%%%%%%%%%%%%%
%%%%%%%%%%%%%%%%%%%%%%%%%%%%%%%%%%%%%%%%%%%%%%%%%%%%%%%%%%%%%%%%%%%%%%%%%%%%%%%%
\textbf{Parallel Computation}
\label{parallel-computation}

In addition to the Cache component of XZES40-Transformer (\ref{cache}), the application will also carry out certain computations in parallel to further leverage the computing resources available to it and compile documents even faster.

%%%%%%%%%%%%%%%%%%%%%%%%%%%%%%%%%%%%%%%%%%%%%%%%%%%%%%%%%%%%%%%%%%%%%%%%%%%%%%%%
\textbf{Context}

The Parallel Computation component of the application carries out the following operations in parallel to speed up documentation transformations.

\textbf{Document Parsing}

This will be carried out in parallel.
These operations are logically independent so they can be carried out simultaneously without affecting data integrity.

\textbf{User Requests} 

This will also be carried out in Parallel, handled by Apache.

%%%%%%%%%%%%%%%%%%%%%%%%%%%%%%%%%%%%%%%%%%%%%%%%%%%%%%%%%%%%%%%%%%%%%%%%%%%%%%%%
\textbf{Dependencies}

The Parallel Computation component of the application will be carried out at the high level by Apache delegating parsing jobs, see the \gls{web-api} subsubsection for more information \ref{web-api}.
the application will also carry out parallel computation internally (C++) using the MPI library.

%%%%%%%%%%%%%%%%%%%%%%%%%%%%%%%%%%%%%%%%%%%%%%%%%%%%%%%%%%%%%%%%%%%%%%%%%%%%%%%%
\textbf{Interactions}

Our MPI-based thread computing will operate mostly autonomously, except when putting data into and fetching data from the application Cache (\ref{cache}).
With the exception of interacting with the Cache each parsing thread will not interact with other internal components of the application.

%%%%%%%%%%%%%%%%%%%%%%%%%%%%%%%%%%%%%%%%%%%%%%%%%%%%%%%%%%%%%%%%%%%%%%%%%%%%%%%%
\textbf{Algorithms}

One major concern with handling a cache by multiple threads and processes in parallel is avoiding data corruption.
This is not a problem we have yet solved and further revisions of this document will elaborate on how we will handle this dilemma.

%%%%%%%%%%%%%%%%%%%%%%%%%%%%%%%%%%%%%%%%%%%%%%%%%%%%%%%%%%%%%%%%%%%%%%%%%%%%%%%%
\textbf{Interfaces}

When documents are being parsed each parsed document will spawn it's own thread.
This functionality is in the Document class component of the application in subsubsection \ref{parser}.

%%%%%%%%%%%%%%%%%%%%%%%%%%%%%%%%%%%%%%%%%%%%%%%%%%%%%%%%%%%%%%%%%%%%%%%%%%%%%%%%
%%%%%%%%%%%%%%%%%%%%%%%%%%%%%%%%%%%%%%%%%%%%%%%%%%%%%%%%%%%%%%%%%%%%%%%%%%%%%%%%
\subsubsection{Daemon}
\label{daemon}

%%%%%%%%%%%%%%%%%%%%%%%%%%%%%%%%%%%%%%%%%%%%%%%%%%%%%%%%%%%%%%%%%%%%%%%%%%%%%%%%
\textbf{Context}

Because the application needs to be available continuously, and because it needs to store objects in an in-memory Cache, the application is daemonized.
This means that it continues to run in a suspended state when it is not actively performing document transformations.
When it does receive an incoming request it spawns a thread, does the transformation, and closes the thread.
This allows multiple document transformations to happen in parallel conveniently.

%%%%%%%%%%%%%%%%%%%%%%%%%%%%%%%%%%%%%%%%%%%%%%%%%%%%%%%%%%%%%%%%%%%%%%%%%%%%%%%%
\textbf{Composition}

The daemon is composed of two main components:

\begin{itemize}
  \item {
    A parent thread which will wait for a signal to handle a request.
    This signal is accompanied with some context for what task needs to be performed, i.e., a document transformation.
  }
  \item {
    A child thread is spawned for each incoming signal.
    The child process exits at the end of it's transformation after returning the transformed document or an error which is propagated up through the parent process to the signaling process.
  }
\end{itemize}

%%%%%%%%%%%%%%%%%%%%%%%%%%%%%%%%%%%%%%%%%%%%%%%%%%%%%%%%%%%%%%%%%%%%%%%%%%%%%%%%
\textbf{Interfaces}

The interface to the daemon is the same as that of the daemon.
A CLI on the host running the daemon will accept an XML and XSL document as input, pass these to the daemon, and respond with an error or transformed document.

%%%%%%%%%%%%%%%%%%%%%%%%%%%%%%%%%%%%%%%%%%%%%%%%%%%%%%%%%%%%%%%%%%%%%%%%%%%%%%%%
%%%%%%%%%%%%%%%%%%%%%%%%%%%%%%%%%%%%%%%%%%%%%%%%%%%%%%%%%%%%%%%%%%%%%%%%%%%%%%%%
%%%%%%%%%%%%%%%%%%%%%%%%%%%%%%%%%%%%%%%%%%%%%%%%%%%%%%%%%%%%%%%%%%%%%%%%%%%%%%%%
\subsubsection{User Interface}
\label{user-interface}

The Document Transformer is fine and great, but without a user interface it's not useful.
The following two subsubsections, the \gls{web-api} (\ref{web-api}), Website interface (\ref{website}), and \gls{cli} (\ref{cli}) together outline the ways users will will interact with the system described in the previous subsubsection.

%%%%%%%%%%%%%%%%%%%%%%%%%%%%%%%%%%%%%%%%%%%%%%%%%%%%%%%%%%%%%%%%%%%%%%%%%%%%%%%%
%%%%%%%%%%%%%%%%%%%%%%%%%%%%%%%%%%%%%%%%%%%%%%%%%%%%%%%%%%%%%%%%%%%%%%%%%%%%%%%%
\textbf{\gls{web-api}}
\label{web-api}

The \gls{web-api} is the standardized interface between the user interfaces and an instance of the application running on a host, communicating over \gls{http}.

%%%%%%%%%%%%%%%%%%%%%%%%%%%%%%%%%%%%%%%%%%%%%%%%%%%%%%%%%%%%%%%%%%%%%%%%%%%%%%%%
\textbf{Context}

The users for the \gls{web-api} are those who use the application from the \gls{cli} (\ref{cli}) and (Website \ref{website}).
Both interfaces interact with the XZES40-Transformer host via standard \gls{http} request methods.
But nobody should use the \gls{api} directly as a \gls{ui} is much easier than crafting an \gls{http} POST request.

%%%%%%%%%%%%%%%%%%%%%%%%%%%%%%%%%%%%%%%%%%%%%%%%%%%%%%%%%%%%%%%%%%%%%%%%%%%%%%%%
\textbf{Composition}

The \gls{web-api}is composed of the following components:

\begin{enumerate}
  \item  An \gls{apache} runs on the remote host.\cite{apache-server}
  \item The server manages a \gls{python} \gls{cgi} script which handles accepting requests and sending responses.
  \item The \gls{python} script calls the XZES40-Transformer application locally, passing input documents from a POST request and sending response files via the \gls{cgi} interface.
\end{enumerate}

%%%%%%%%%%%%%%%%%%%%%%%%%%%%%%%%%%%%%%%%%%%%%%%%%%%%%%%%%%%%%%%%%%%%%%%%%%%%%%%%
\textbf{Dependencies}

The \gls{web-api} internally depends on an XZES40-Transformer binary which accepts an input \gls{xml} file, and input \gls{xslt} file, and writes a transformed file to disk in a predictable location.
The binary should also exit with predictable exit codes to communicate any errors or successes.

%%%%%%%%%%%%%%%%%%%%%%%%%%%%%%%%%%%%%%%%%%%%%%%%%%%%%%%%%%%%%%%%%%%%%%%%%%%%%%%%
\textbf{Interfaces}

The \gls{web-api} communicates with remote clients via a standard \gls{http} \gls{api}.
Below are the requests a client can send and the possible responses:

\textbf{POST /api/}

This is a request sent to the \gls{api} endpoint (/api/) with the intention of getting two input files transformed into a new document.
This can respond in the following ways.

\begin{description}
  \item {
      \textbf{200 OK}  Means the transformation was successful.
        This response includes a body containing the transformed file and a \gls{uri} to re-download the response file.
    }
    \item {
      \textbf{400 USER ERROR} Means that the user sent malformed documents.
        This can include a document which does not follow the \gls{xml}/\gls{xslt} standard to a document which does not have a readable character encoding.
        This response includes a body containing an appropriately specific error.
    }
    \item {
      \textbf{500 SERVER ERROR} Means that the server experienced an internal error while processing the request.
         This includes fatal XZES40-Transformer errors.
        This response includes a body containing an appropriately specific error.
    }
\end{description}

The POST request is a request with a Form containing input documents in fields titled \gls{xml} and \gls{xslt} for an \gls{xml} 1.0 formatted document and an \gls{xslt} 1.0 formatted document respectively.

\textbf{GET /api/}

This is a request sent to the \gls{api} endpoint (/api/) with the intention of getting a status of the server. This can respond in the following ways:

\begin{description}
  \item \textbf{200 OK} Means the \gls{web-api} endpoint is active and functioning correctly.
    \item \textbf{404 NOT FOUND} Means that the \gls{web-api} endpoint is not configured correctly or the user is accessing a page which is not available.
    \item \textbf{503 SERVICE UNAVAILABLE} Means that the \gls{web-api} is setup correctly, but the application on the remote host is not operating correctly.
\end{description}

The GET request is an empty GET request to the servers ``/api/'' endpoint.

\textbf{\gls{python} Interfaces}

The following interfaces are used for implementing the above \gls{http} interfaces.

\textbf{process\_request(http-request req)}

This method is used to receive an \gls{http} \gls{api} request.
It does this by reading the request header and deciding based on that to carry out one of the above responses.
In the case of a POST request it uses an ``exec'' call to the local XZES40-Transformer application binary and responds with the error / output file of that application.

%%%%%%%%%%%%%%%%%%%%%%%%%%%%%%%%%%%%%%%%%%%%%%%%%%%%%%%%%%%%%%%%%%%%%%%%%%%%%%%%
\textbf{Resources}

The \gls{web-api} has the following external dependencies:

\begin{itemize}
  \item The \textbf{\gls{apache}} is used to process \gls{http} requests by running the \gls{python} \gls{cgi} script.
    \item \textbf{\gls{python} 2.7+} is the programming language the \gls{web-api} is written in.
    \item \textbf{cgi} is a \gls{python} library for processing web requests in \gls{python}.
    \item \textbf{cgitb} is a \gls{python} library for developing with the cgi library.
    \item \textbf{mod\_wsgi} is an Apache module for interacting with \gls{python}.
    \item \textbf{mod\_python} is another Apache module for interacting with \gls{python}.
\end{itemize}

They should be installed via a system package manager or the \gls{python} package manager (whichever is appropriate) for the \gls{web-api} to operate correctly.

%%%%%%%%%%%%%%%%%%%%%%%%%%%%%%%%%%%%%%%%%%%%%%%%%%%%%%%%%%%%%%%%%%%%%%%%%%%%%%%%
%%%%%%%%%%%%%%%%%%%%%%%%%%%%%%%%%%%%%%%%%%%%%%%%%%%%%%%%%%%%%%%%%%%%%%%%%%%%%%%%
\textbf{Website}
\label{website}

The website interface is the major user interface for the XZES40 application.
Website interface is a \gls{gui} interface for user.

%%%%%%%%%%%%%%%%%%%%%%%%%%%%%%%%%%%%%%%%%%%%%%%%%%%%%%%%%%%%%%%%%%%%%%%%%%%%%%%%
\textbf{Context}

The Website interface is common interface for application.
User can uses website to complete request with few of instructions.
User can upload the \gls{xml}/\gls{xslt} files that they want, and the website will give the feed back.

%%%%%%%%%%%%%%%%%%%%%%%%%%%%%%%%%%%%%%%%%%%%%%%%%%%%%%%%%%%%%%%%%%%%%%%%%%%%%%%%
\textbf{Composition}

The web interface to access the application in a browser.
Here is components of the \gls{web-ui}.

\begin{itemize}
    \item An upload field for the \textbf{\gls{xml}} file.
    \item An upload field for the \textbf{\gls{xslt}} file.
    \item A button for user send the request to server via \textbf{POST} method.
    \item {
       After user sends a request to server the website will respond with a message to user.
        This message may be error or successfully upload or generate download link.
    }
\end{itemize}

%%%%%%%%%%%%%%%%%%%%%%%%%%%%%%%%%%%%%%%%%%%%%%%%%%%%%%%%%%%%%%%%%%%%%%%%%%%%%%%%
\textbf{State Dynamics}

There are few of states for website.

\begin{itemize}
    \item If user upload bad files the website gives feedback that the files were bad.
    \item If user upload files and server transforming the documents the website gives the feedback that user upload files successfully.
    \item If user uploads files and server is down the website gives the warning that transformation service is not working.
    \item If user uploads good files and the server successfully transforms the documents the user is given a download link to the new file.
\end{itemize}

%%%%%%%%%%%%%%%%%%%%%%%%%%%%%%%%%%%%%%%%%%%%%%%%%%%%%%%%%%%%%%%%%%%%%%%%%%%%%%%%
\textbf{Interactions}

The initialization status of website is waiting for user upload files.
After user select flies that they want upload and click the submit button.
Website will send the request to server via the \gls{web-api}.
There are many possible feedback as following.

\begin{itemize}
    \item If the files uploaded are bad the website pops up a warning about the malformed files.
    \item  If the user uploads both an \gls{xml} and \gls{xslt} file the user is alerted that the request is good to go and that the website will generate a download link for them.
    \item If the connection is broken the website will give them a warning that user should check the connection between client and server.
\end{itemize}

%%%%%%%%%%%%%%%%%%%%%%%%%%%%%%%%%%%%%%%%%%%%%%%%%%%%%%%%%%%%%%%%%%%%%%%%%%%%%%%%
\textbf{Interfaces}

The Web interface is XZES40 main interface for user.
The web interface asks user to upload \gls{xml} files and \gls{xslt} files, and gives the feedback of result.
Here is the diagram for website interface.

% \textbf{Web interface Diagram}

%%%%%%%%%%%%%%%%%%%%%%%%%%%%%%%%%%%%%%%%%%%%%%%%%%%%%%%%%%%%%%%%%%%%%%%%%%%%%%%%
%%%%%%%%%%%%%%%%%%%%%%%%%%%%%%%%%%%%%%%%%%%%%%%%%%%%%%%%%%%%%%%%%%%%%%%%%%%%%%%%
\textbf{\gls{cli}}
\label{cli}

The \gls{cli} is used to interact with the system via a text-based terminal/shell interface.

%%%%%%%%%%%%%%%%%%%%%%%%%%%%%%%%%%%%%%%%%%%%%%%%%%%%%%%%%%%%%%%%%%%%%%%%%%%%%%%%
\textbf{Context}

The users of this interface are individuals who either prefer the \gls{cli} over a web interface or for testing purposes as automating tasks with the \gls{cli} is very common.

%%%%%%%%%%%%%%%%%%%%%%%%%%%%%%%%%%%%%%%%%%%%%%%%%%%%%%%%%%%%%%%%%%%%%%%%%%%%%%%%
\textbf{Composition}

The \gls{cli}, written in \gls{python}, and will be composed of the following components.

\begin{itemize}
  \item The main function parses command-line arguments specified below in the ``Interfaces'' part of this subsubsection.
    \item A query is built for the server including the \gls{xml}and \gls{xslt} documents to the specified server.
    \item The query is sent to the server in a POST request.
    \item When the response is received either an error message is displayed to the user or a file is saved locally.
\end{itemize}

%%%%%%%%%%%%%%%%%%%%%%%%%%%%%%%%%%%%%%%%%%%%%%%%%%%%%%%%%%%%%%%%%%%%%%%%%%%%%%%%
\textbf{Resources}

The \gls{cli} depends on the following system dependencies:

\begin{itemize}
  \item {Python 2.7+} is the programming language it will be implemented in, so a Python runtime will be necessary for the application to run.
    \item {Requests} is the standard Python library for interacting with servers over \gls{http}.
    \item {A terminal and \gls{unix} compliant shell} will also be necessary to access the application via the \gls{cli}.
\end{itemize}

%%%%%%%%%%%%%%%%%%%%%%%%%%%%%%%%%%%%%%%%%%%%%%%%%%%%%%%%%%%%%%%%%%%%%%%%%%%%%%%%
\textbf{Interfaces}

The \gls{cli} has the following options available at the command-line:
\begin{figure}[H]
    \begin{lstlisting}
$ xzes40
    --server=<server-url[:port]> # API endpoint and port to be used.
    --xml=<input-file>.xml       # Input XML file
    --xslt=<style-sheet>.xslt    # Input XSLT file
    \end{lstlisting}
    \caption{\gls{cli} Flags. All fields encased in chevrons symbols are required. All fields with square brackets are optional.}
\end{figure}

\textbf{Example Use-cases}

The following are examples of the \gls{cli} in use:

\begin{figure}[H]
    \begin{lstlisting}[caption={}]
$ xzes40 --server=http://xzes40.example.com:8080 \
         --xml=input-file.xml \
         --xslt=style.xslt
Sending input-file.xml and style.xslt to the http://xzes40.example.com:8080
Downloading transformed document to ./xzes40-transformer-2016-11-22.xml
    \end{lstlisting}
    \caption{``Happy path'' use-case without specifying return file name.}
\end{figure}


\begin{figure}[H]
   \begin{lstlisting}
$ xzes40 --server=http://xzes40.example.com:8080 \
         --xml=input-file.xml \
         --xslt=style.xslt \
         --output=transformed-doc.xml
Sending input-file.xml and style.xslt to the http://xzes40.example.com:8080
Downloading transformed document to ./transformed-doc.xml
   \end{lstlisting}
   \caption{``Happy path'' use-case with return file name.}
\end{figure}

\begin{figure}[H]
    \begin{lstlisting}
$ xzes40 --server=http://xzes40.example.com:8080 \
         --xml=input-file.xml \
         --xslt=style.xslt \
Sending input-file.xml and style.xslt to the http://xzes40.example.com:8080
Server responded with error. One of your documents is malformed.
    \end{lstlisting}
    \caption{``Un-happy path'' with malformed document.}
\end{figure}


``Un-happy path'' with bad host.
\begin{figure}[H]
  \begin{lstlisting}
$ xzes40 --server=http://fakesite.com:8080 \
         --xml=input-file.xml --xslt=style.xslt \
Sending input-file.xml and style.xslt to the http://xzes40.example.com:8080
Unable to reach server. Is the host/port correct?
  \end{lstlisting}
  \caption{``Un-happy path'' with bad host.}
\end{figure}

\begin{figure}[H]
  \begin{lstlisting}
$ xzes40  --xml=input-file.xml--xslt=style.xslt
Please specify a host
  \end{lstlisting}
  \caption{``Un-happy path'' with no host.}
\end{figure}

\begin{figure}[H]
  \begin{lstlisting}
$ xzes40  --server=http://xzes40.example.com:8080
Please specify an input xml and/or xslt document.
  \end{lstlisting}
  \caption{``Un-happy path'' with missing input document.}
\end{figure}

%%%%%%%%%%%%%%%%%%%%%%%%%%%%%%%%%%%%%%%%%%%%%%%%%%%%%%%%%%%%%%%%%%%%%%%%%%%%%%%%
%%%%%%%%%%%%%%%%%%%%%%%%%%%%%%%%%%%%%%%%%%%%%%%%%%%%%%%%%%%%%%%%%%%%%%%%%%%%%%%%
%%%%%%%%%%%%%%%%%%%%%%%%%%%%%%%%%%%%%%%%%%%%%%%%%%%%%%%%%%%%%%%%%%%%%%%%%%%%%%%%
\subsubsection{System Requirements}
\label{system-requirements}

The XZES40-Transformer will be required to work on the Debian Linux system.
Once development on Debian is completed the application will be ported to other \gls{os} platforms.
Our team will release \gls{debian}, \gls{centos}, \gls{bsd} and \gls{windows} packages.
  
%%%%%%%%%%%%%%%%%%%%%%%%%%%%%%%%%%%%%%%%%%%%%%%%%%%%%%%%%%%%%%%%%%%%%%%%%%%%%%%%
%%%%%%%%%%%%%%%%%%%%%%%%%%%%%%%%%%%%%%%%%%%%%%%%%%%%%%%%%%%%%%%%%%%%%%%%%%%%%%%%
\textbf{Installation Packages}
\label{installation-packages}

For installation convenience the XZES40 project will provide a \gls{linux}, \gls{bsd}, and \gls{windows} installation packages.

%%%%%%%%%%%%%%%%%%%%%%%%%%%%%%%%%%%%%%%%%%%%%%%%%%%%%%%%%%%%%%%%%%%%%%%%%%%%%%%%
\textbf{Context}

The user can directly download installation packages through the internet and install the XZES40-Transformer on their local systems.
We will create the Debian package initially.
After this, we will create the \gls{centos}, \gls{bsd}, and \gls{windows} packages.
We will upload those packages to the internet and the users can directly download the packages in their operation system.

%%%%%%%%%%%%%%%%%%%%%%%%%%%%%%%%%%%%%%%%%%%%%%%%%%%%%%%%%%%%%%%%%%%%%%%%%%%%%%%%
\textbf{Resources}

The following tools will be used to create the installation packages. 
\begin{figure}[H]
  \begin{center}                                                                       
    \begin{tabular}{ | p{2.5cm} | p{5cm} | p{5cm} | }
    \hline
      Packages & Tools & Description \\ \hline
      \gls{linux} \& \gls{bsd} Packages  & 
      \begin{itemize}
      \item \gls{fpm}
      \end{itemize} &
      \begin{itemize}
      \item Translates packages from one format to another
      \item Allows re-use of other system's packages
      \end{itemize} \\ \hline

      \gls{windows} Packages & 
      \begin{itemize}
      \item \gls{wix}
      \end{itemize} &
      \begin{itemize}
      \item It is a open source project.
      \item It is more stable and security than other tools.
      \item It has steed
      \end{itemize} \\ \hline

    \end{tabular}
\end{center}
\caption{Installation Packages Resources}
\end{figure}
%%%%%%%%%%%%%%%%%%%%%%%%%%%%%%%%%%%%%%%%%%%%%%%%%%%%%%%%%%%%%%%%%%%%%%%%%%%%%%%%
%%%%%%%%%%%%%%%%%%%%%%%%%%%%%%%%%%%%%%%%%%%%%%%%%%%%%%%%%%%%%%%%%%%%%%%%%%%%%%%%
\textbf{\gls{os-api}}
\label{os-agnostic-api}

The \gls{os-api} will be the programs interface with the \gls{os}.
It will be created to make porting the application easier. 

%%%%%%%%%%%%%%%%%%%%%%%%%%%%%%%%%%%%%%%%%%%%%%%%%%%%%%%%%%%%%%%%%%%%%%%%%%%%%%%%
\textbf{Logical}

Any operating system specific operation will be wrapped by the \gls{api}.

%%%%%%%%%%%%%%%%%%%%%%%%%%%%%%%%%%%%%%%%%%%%%%%%%%%%%%%%%%%%%%%%%%%%%%%%%%%%%%%%
\textbf{Interactions}

The application will interact with the host system via an \gls{os-api}.
This means that all operating system specific operations (e.g., read, write, seek, etc) will be done via an \gls{api}.
When the application is compiled on a new target platform (e.g., \gls{linux}, \gls{bsd}, \gls{windows}) a new platform \gls{api} must be created for compatibility.

%%%%%%%%%%%%%%%%%%%%%%%%%%%%%%%%%%%%%%%%%%%%%%%%%%%%%%%%%%%%%%%%%%%%%%%%%%%%%%%%
\textbf{Interfaces}

The \gls{os-api} interface is for the developer.
This interface performs operations by asking the \gls{os-api} to carry out the task and that request is translated to the system-specific system-call. 

%%%%%%%%%%%%%%%%%%%%%%%%%%%%%%%%%%%%%%%%%%%%%%%%%%%%%%%%%%%%%%%%%%%%%%%%%%%%%%%%
%%%%%%%%%%%%%%%%%%%%%%%%%%%%%%%%%%%%%%%%%%%%%%%%%%%%%%%%%%%%%%%%%%%%%%%%%%%%%%%%
\textbf{Performance Benchmark}
\label{performance-benchmark}

%%%%%%%%%%%%%%%%%%%%%%%%%%%%%%%%%%%%%%%%%%%%%%%%%%%%%%%%%%%%%%%%%%%%%%%%%%%%%%%%
\textbf{Context}

We will run test against similar application to determine how fast XZES40 should be.
Throughout development we will put our application through the same paces and compare which is faster.

%%%%%%%%%%%%%%%%%%%%%%%%%%%%%%%%%%%%%%%%%%%%%%%%%%%%%%%%%%%%%%%%%%%%%%%%%%%%%%%%
\textbf{Resources}

We will be bench-marking these applications specifically for the following reasons.

\begin{figure}[H]
  \centering
  \begin{tabular}{ | l | p{10cm} |}
    \hline
    Technology & Description  \\ \hline
    \gls{xalan} \gls{cli}&
    \begin{itemize}
      \item \gls{xalan} uses \gls{xerces} to parse \gls{xml} documents and \gls{xslt}.
      \item The project provides an open source \gls{cli} program to test the project libraries.
      \item Free and Open Source
    \end{itemize} \\ \hline
    Altova &
    \begin{itemize}
      \item To meet industry demands for an ultra-fast processor.
      \item It offers powerful, flexible options for developers including cml, python.
      \item Superior error reporting capabilities include reporting of multiple errors, detailed error descriptions.
    \end{itemize} \\ \hline
  \end{tabular}
  \caption{Bench-marking Resources}
\end{figure}


%%%%%%%%%%%%%%%%%%%%%%%%%%%%%%%%%%%%%%%%%%%%%%%%%%%%%%%%%%%%%%%%%%%%%%%%%%%%%%%%
%%%%%%%%%%%%%%%%%%%%%%%%%%%%%%%%%%%%%%%%%%%%%%%%%%%%%%%%%%%%%%%%%%%%%%%%%%%%%%%%
%%%%%%%%%%%%%%%%%%%%%%%%%%%%%%%%%%%%%%%%%%%%%%%%%%%%%%%%%%%%%%%%%%%%%%%%%%%%%%%%
\subsubsection{Design Rationale}

This subsubsection explains why certain design decisions were made and ``connects the pieces'' of the application.

%%%%%%%%%%%%%%%%%%%%%%%%%%%%%%%%%%%%%%%%%%%%%%%%%%%%%%%%%%%%%%%%%%%%%%%%%%%%%%%%
%%%%%%%%%%%%%%%%%%%%%%%%%%%%%%%%%%%%%%%%%%%%%%%%%%%%%%%%%%%%%%%%%%%%%%%%%%%%%%%%
\textbf{Cache Decisions}

The reason the Cache uses an in-memory system rather than a database or a file-based cache is purely for performance reasons.
Using an in-memory cache over a file-based one should yield faster performance.
As a compromise the cache is dumped to a file periodically to save the state of the cache in case the application daemon (which holds the cache) is restarted.

This may cause problems as the service may run out of memory.
In an attempt to mitigate this a Garbage Collector may be built which is triggered when a certain percentage of the application's allocated memory is used, or on certain time intervals.
 
%%%%%%%%%%%%%%%%%%%%%%%%%%%%%%%%%%%%%%%%%%%%%%%%%%%%%%%%%%%%%%%%%%%%%%%%%%%%%%%%
%%%%%%%%%%%%%%%%%%%%%%%%%%%%%%%%%%%%%%%%%%%%%%%%%%%%%%%%%%%%%%%%%%%%%%%%%%%%%%%%
\textbf{\gls{web-api} Decisions}

The \gls{web-api} connects the application to the outside world in an ideally simple to implement and quick to operate fashion.
The application uses Apache to handle incoming requests, these are passed to a \gls{cgi} script which could be written in anything, we chose \gls{python} because it is easy to write, maintain, and is well supported.

\gls{python} is called by \gls{apache} which then in turn, based on the request, either returns an application status or calls the XZES40 application locally.
In this way \gls{python} with \gls{apache} is a simple, well supported set of tools which expose the application to the outside world.

These tools were not necessarily chosen for their speed, and so a redesign may be necessary if they create a bottle-neck in the request pipeline.

%%%%%%%%%%%%%%%%%%%%%%%%%%%%%%%%%%%%%%%%%%%%%%%%%%%%%%%%%%%%%%%%%%%%%%%%%%%%%%%%
%%%%%%%%%%%%%%%%%%%%%%%%%%%%%%%%%%%%%%%%%%%%%%%%%%%%%%%%%%%%%%%%%%%%%%%%%%%%%%%%
\textbf{Packaging Decisions}

This document outlines the use of \gls{fpm} to create its \gls{linux} and \gls{unix} packages.
This decision was made for convenience and to allow for quick iteration on the package.
In doing research there \textit{are} other tools which can be used to create packages on \gls{centos}, \gls{debian}, and \gls{bsd} but they are very system specific and so would make iterating on the package very difficult, and updates to the software a pain to package.
Using \gls{fpm} we can even automate the build of packages on our \gls{unix}-like systems.

%%%%%%%%%%%%%%%%%%%%%%%%%%%%%%%%%%%%%%%%%%%%%%%%%%%%%%%%%%%%%%%%%%%%%%%%%%%%%%%%
%%%%%%%%%%%%%%%%%%%%%%%%%%%%%%%%%%%%%%%%%%%%%%%%%%%%%%%%%%%%%%%%%%%%%%%%%%%%%%%%
\textbf{Bench-marking decisions}

To demonstrate the competitiveness of the XZES40-Transformer application we chose to compare the performance of this application against the Xalan C++ and the Altova application.

This feels like a fair and balanced comparison as we hope to be competitive with the Altova application but need to ensure we are at least better than the Xalan C++.


\subsection{Changes}
%%%%%%%%%%%%%%%%%%%%%%%%%%%%%%%%%%%%%%%%%%%%%%%%%%%%%%%%%%%%%%%%%%%%%%%%%%%%%%%%
% A discussion of what had to change over the course the year.
%%%%%%%%%%%%%%%%%%%%%%%%%%%%%%%%%%%%%%%%%%%%%%%%%%%%%%%%%%%%%%%%%%%%%%%%%%%%%%%%


\section{Technology review}
%%%%%%%%%%%%%%%%%%%%%%%%%%%%%%%%%%%%%%%%%%%%%%%%%%%%%%%%%%%%%%%%%%%%%%%%%%%%%%%%
% Introduction; what a technology review is.
%%%%%%%%%%%%%%%%%%%%%%%%%%%%%%%%%%%%%%%%%%%%%%%%%%%%%%%%%%%%%%%%%%%%%%%%%%%%%%%%

In creating a piece of software it is best to look around, see what exists, and make an informed decisions about the tools you should use to carry out the project.
This section outlines those technologies investigated and the reason the tools chosen were used.

\subsection{Preliminary technology review}
%%%%%%%%%%%%%%%%%%%%%%%%%%%%%%%%%%%%%%%%%%%%%%%%%%%%%%%%%%%%%%%%%%%%%%%%%%%%%%%%
% Your tech review, in its original form.
%%%%%%%%%%%%%%%%%%%%%%%%%%%%%%%%%%%%%%%%%%%%%%%%%%%%%%%%%%%%%%%%%%%%%%%%%%%%%%%%

This is the preliminary technology review with minimal edits.
It includes the original references page for authenticity.

\subsubsection{Introduction}

The following tables illustrates who is responsible for each component of the application.

Required functionality:

\begin{center}
    \begin{tabular}{ | l | p{10cm} |}
    \hline
    Section & Author \\ \hline
    Research and Benchmarking & Zixun Lu \\ \hline
    XML/XSLT Document Transformation & Shuai Peng \\ \hline
    XML/XSLT Document Parsing & Elijah C. Voigt \\ \hline
    XML/XSLT Document Caching & Shuai Peng \\ \hline
    XML/XSLT Document Parallel computation & Zixun Lu \\ \hline
    Web API & Elijah C. Voigt \\ \hline
    Web Interface & Shuai Peng \\ \hline
    \end{tabular}
\end{center}

Stretch goal functionality:

\begin{center}
    \begin{tabular}{ | l | p{10cm} |}
    \hline
    Linux Package & Zixun Lu \\ \hline
    Command Line Interface& Elijah C. Voigt \\ \hline
    Windows Packae & Shuai Peng \\ \hline
    BSD Package & Elijah C. Voigt \\ \hline
    \end{tabular}
\end{center}

\tableofcontents

\clearpage

%%%%%%%%%%%%%%%%%%%%%%%%%%%%%%%%%%%%%%%%%%%%%%%%%%%%%%%%%%%%%%%%%%%%%%%%%%%%%%%%

\newpage

\printbibliography

% Section One: Research and Benchmarking
% Owned by Zixun Lu 
%\subsubsubsubsection{Research and Benchmarking}
%The first application necessary for XML data processing are XML parsers and XML validators. The key aim is to check correctness of the input data, i.e. their conformance to either W3C recommendations or respective XML schemes. Hence, the benchmarks usually involve sets of correct and incorrect data and the goal is to test whether the application under test recognizes them correctly.
%\paragraph{XML Conformance Test Suites}
%Binary tests contain a set of documents of one of the following categories: valid documents, invalid documents, non-well-formed documents, well-formed errors tied to external entity and documents with optional errors. Depending on the category, the tested parser must either accept or reject the document correctly (therefore, the tests are called binary). The expected behavior naturally differs if the tested parser is validating or non-validating. On the other hand, the output tests enable to test whether the respective applications report information as required by the recommendation. Again, validating processors are required to report more information than non-validating ones.
%\paragraph{}

%\paragraph{}
\subsubsection{Research and Benchmarking}

\paragraph{Options}

\paragraph{Goals for use in design}

Our team will do the research and benchmarking to grantee that our application is fast enough.

\paragraph{Criteria being evaluated}

We will test some number of requests against a comparable to find which transformers use the time less.
Throughout development we will put our application through the same paces and compare which is faster.

\paragraph{Comparison breakdown}

\begin{center}
  \begin{tabular}{ | l | p{10cm} |}
    \hline
    Technology & Description  \\ \hline

    Xalan CLI \cite{Xalan-C} &
    \begin{itemize}
      \item Xalan-C++ uses Xerces-C++ to parse XML documents and XSL stylesheets.
      \item The project provides an open source CLI program to test the project libraries.
      \item Free and Open Source
      \item It works on the Debian operation system. 
    \end{itemize}\\ \hline

    Altova \cite{Altova} &
    \begin{itemize}
      \item To meet industry demands for an ultra-fast processor.
      \item It offers powerful, flexible options for developers including cml, python.
      \item Superior error reporting capabilities include reporting of multiple errors, detailed error descriptions.
      \item It only works in the Windows operation system. 
    \end{itemize} \\ \hline

  \end{tabular}
\end{center}

\paragraph{Discussion}

Xerces is a simple CLI application developed by the Xerces project to test the library.
This is very similar to our program as it is open source, uses the same libraries, but lacks the caching we will implement.

RaptorXML is built from the ground up to be optimized for the latest standard and parallel computing environments.
It is proprietary tool which we may try to out-perform as a stretch goal, but to start with out application will not try to out-perform.

\paragraph{Selection}

We will compare our application to the Xalan-C CLI as it is the closest competitor to our application.

\subsubsection{XML/XSLT Document Parsing (Elijah C. Voigt)}

XML document compilation is the process by which an XML formatted document is taken and parsed into an in-memory object.
The library we will be using, as requested by or client, will be the Xerces-C/C++ XML parser library; this review will evaluate that library.
The other point of wiggle-room we have is in what way we parse and store the document in memory.

\paragraph{Options}

For this requirement we will review the C/C++ library that we were requested to use by our client, and two XML parsing techniques we may use.
The two techniques for parsing an XML document include SAX and DOM, the library we need to use is able to perform both methods.
These are both designed with specific use-cases in mind, however the method we ought to use is not so simply chosen.

\paragraph{Goals for use in design}

With the goal of optimizing peformance in mind we will choose the parsing option which is fastest for our application.
This is not as simple as choosing the one which is known for being fastest, because we will also be caching documents and performing parallel computations which may pull the needle toward one method of compiliation over another.

\paragraph{Criteria being evaluated}

We will be evaluating the parsing method which is likely to perform best in our application given that we will be caching documents and compiling them in parallel.

\paragraph{Comparison breakdown}

\begin{center}
  \begin{tabular}{ | l | p{10cm} | }
    \hline
    Technology & Description  \\ \hline
    Xerces-C/C++ \cite{xerces} &
    \begin{itemize}
      \item Requested to be used by our client.
      \item Implements DOM parsing.
      \item Implements SAX parsing.
      \item Feature Rich.
    \end{itemize}\\ \hline
    DOM parsing \cite{dom-vs-sax} &
    \begin{itemize}
      \item Memory intensive.
      \item Ideal for carrying out many operations on a document.
    \end{itemize}\\ \hline
    SAX parsing \cite{dom-vs-sax} &
    \begin{itemize}
      \item Memory light.
      \item Event-based processing.
      \item Ideal for minimal transformations.
    \end{itemize}\\ \hline
  \end{tabular}
\end{center}

\paragraph{Discussion}

DOM parsing is know for being memory intensive and slower in simple cases.
It is also know for being ideal in applications like updating a web-page where many operations happen to a document.
It is not necessarily ideal for simple document transformation, but does produce a complete parsed object which we can cache easily.

SAX parsing is known for being better than DOM parsing in that it parses a document in an event based method, giving it a smaller memory footprint and shorter time-to-transform.
Simple transformations are common use-cases for SAX parsing.
SAX does not produce an object model and instead uses callbacks to perform document transformations.

\paragraph{Selection}

We will be using the Xerces-C/C++ library to accomplish the task of XML document parsing.
It is feature rich enough to give us leeway in our development where we need it
The library is also lean enough that it should not affect performance negatively.

As for choosing between SAX parsing vs DOM parsing, we will most likely choose DOM parsing since it fits our application's caching requirement well.
DOM parsing produces an easily cached tree object which we can store and retrieve for later operations.
SAX parsing is faster than DOM parsing, but the time saved by using a cached object instead of re-parsing an XML/XSLT document dependency will likely outweigh the benefits of SAX parsing.
It may be worth-while to investigate using SAX parsing early on in development if we find a notable performance boost, so some amount of SAX proof of concept work should be done for the sake of being thorough, but DOM parsing will be the targeted document parsing method.

% Section Two: Document Transformation
% Owned by Shuai Peng
\subsubsection{XML/XSLT Document Transformation}

\paragraph{Option}

There are three options for the XML/XSLT Document Transformation.
The first option is Xalan-C++, and the second option is Saxon C, the third option is Sablotron.

\paragraph{Goals for use in design}

The major function of XZES40-Transformer is XML/XSLT document transforming.
Apache foundation provide many ways to achieve this function.

\paragraph{Criteria being evaluated}

The XML/XSLT document transformation is our major function for XZES40-Transformer application.
We want this transformation compatible with good parse tools, and version of XSLT and XPath.

\paragraph{Comparison breakdown}

The first option is Xalan-C++, this technology is required by our client, and this technology is supported by Apache.
Xalan-C++ is an XSLT processor for transforming XML documents into HTML, text, or other XML document types.
Xalan-C++ contain the XercesC tools, so we don't need consider the parse tools.

The second option is Saxon C, which is Saxon company project.
It performs the same operations as Xalan-C++, but it supports different version of XSLT and Xpath.

The third option technology is Sablotron.
It used same version of XSLT and Xpath with Xalan-C++, but it designed to be as small program.

\begin{table}[H]
  \begin{center}
    \begin{tabular}{ | l | p{10cm} |}
      \hline
      Technology & Description  \\ \hline

      Xalan-C++ \cite{xalan} &
      \begin{itemize}
        \item Xalan-C++ is open source project developed by Apache. It is implemented by XSLT version 1.0 and XPath version 1.0.
        \item Xalan-C++ uses Xerces-C++ to parse XML documents and XSL style sheets.
      \end{itemize}\\ \hline

      Saxon C \cite{Saxon_c} &
      \begin{itemize}
        \item Saxon C is open source project. It is implemented by XSLT 2.0/3.0 version and XPath version 2.0/3.0.
        \item Saxon C use different parse tools to handle the date.
      \end{itemize}\\ \hline

      Sablotron \cite{Sablotron_intro} &
      \begin{itemize}
        \item Sablotron is open source project by gingerall. It is implemented by XSLT 1.0 and Xpath 1.0.
      \item Sablotron need extra parse tools to complete transformation.
      \end{itemize}\\ \hline
    \end{tabular}
  \end{center}
  \caption{Technology being evaluated for XML document transformation}
\end{table}

\paragraph{Discussion}

The Xalan-C++ is the most powerful and popular XML/XSLT document transformation library and is used in many propetary as well as open source tools.
It a robust implementation of the W3C recommendations for XSL Transformations (XSLT) and the XML Path Language (XPath).
The Xalan-C++ is continuing update, and it is good tools for transforming documents.
However, Xalan-C++ has poor structure API reference, and it makes developer hard to read and understand.
Thus, developer need spend time on doing research with API.

The Saxon C is a good alternative to Xalan-C++.
Saxon C can handle higher version of XSLT and Xpath, however implements different parse tools, so we need find out other technology.

The Sablotron does the same work as the Xalan-C++, but Sablotron is designed to be as compact and portable as possible.
The size of Sablotron is much small than Xalan-C.
However, Sablotron is old and has not been updated it since 2006.

\paragraph{The Best option}

We will choose Xalan-C++ as our solution technology.
The first reason is that Xalan-C is required by our client.
We want to using Xerces-C as our parse tools, so Xalan-C is the best choice.
Xalan-C++ is easy to use with clear example.

The second reason is that Sablotron is not stable.
Even they are open source project, but the community has forgetten it.
Saxon C is a good second option if client require different tools to complete XML/XSLT document transformation.

% Section Four: Document Caching
% Owned by Shuai Peng
\subsubsection{XML/XSLT Document Caching}

\paragraph{Option}

There are three technology options for cache.
The first option is storing cache into memory.
The second option is storing our cache in binary file on disk.
The third option is to create database to handle all of data.

\paragraph{Goals for use in design}

Caching is the "plus one" function of XZES40-Transformer application.
Other similar application compile files each time and this wastes a lot of time and resource.
We will create cache to solve this problem. 

\paragraph{Criteria being evaluated}

We want save the time and resources in our XML transformer, so efficiency is the most element that we consider.
This is not only the speed of reading and writing from the cache, we must also weigh the persistence of the cache to avoid recompiling when the system (application or host) is restarted.

\paragraph{Comparison breakdown}

The first option is storing the cache in-memory, this is the faster and easy way to store cache.

The second option is to create a binary file with the cache.
When we are run our application, we read the cache data in from the cache file into memory.
When the cache is updated it is written back to the original file.

The third option technology is that we create database to handle memory.
This spends time to design and create database.

The fourth option is to use the in-memory key-value store Redis \cite{redis} which is a popular for this kind of task.

\begin{center}
    \begin{tabular}{ | l | p{10cm} |}
    \hline
    Technology & Description  \\ \hline

    Memory &
    \begin{itemize}
      \item Application check data from memory, and put cache into memory.
      \item Retrieving data from memory is the faster way.
    \end{itemize} \\ \hline

    Temporary binary file &
    \begin{itemize}
      \item Application loads binary file when it starts. After we close it, application save binary file in external storage driver.
      \item Loading temporary binary file spend time, so it is slower than memory.
    \end{itemize} \\ \hline

    Database &
    \begin{itemize}
      \item Application access data from database.
      \item It takes time to create and manage a database.
    \end{itemize} \\ \hline

    Redis &
    \begin{itemize}
      \item Objects to be serialized before being cached.
      \item Adding an additional dependency to the project.
      \item Less development time spent daemonizing and designing in-memory caching system.
    \end{itemize} \\ \hline
    \end{tabular}
\end{center}

\paragraph{Discussion}

Storing cache into memory is the most easy way, we just need to allocate memory.
However, the main drawback of this technology is when we close application, all of cache data will be wiped out.
We have to compile file next time when we start running the application.
When we are developing the XZES40-Transformer, we find a tools that KeyList is built in XercesC.
This tools is helpful for managing memory, and it has good data structure.

Creating a binary file can avoid losing cache data, but it spends time to load file into memory when application starts.

Creating database is bad option for XZES40-Transformer application because access database spend resource, and it waste time to search cache data.

Redis is an appealing idea, but because it is an additional dependency, and serializing our parsed documents is not something our client wants our application to do, we are not able to pursue it.
If our application had slightly different requirements this may be a viable option.

\paragraph{Selection}

The best option technology is that storing cache into memory.
Although it will lost data after close application, it save the time, and it the faster way.
We may add a 'backup cache' solution to make this the best of both worlds, restoring from the backup when the system restarts but working mostly in-memory.
And we will using KeyList, because it is built in XercesC.
We can just add API, and easy to control the cache storing.

% Section Five: Prallel Transformation 
% Owned by Zixun Lu 
\subsubsection{Parallel Document Transformation}

\paragraph{Options}

\paragraph{Goals for use in design}

XZES40-Transformer will be increased performance by using parrallel computation.
The parallel processing of the containment queries against an XML document utilizes parallel variants of the serial algorithm.
There are two ways to do the prallel computation.


\paragraph{Criteria being evaluated}

XZES40-Transformer use the parallel computation method do the same processing and operating in parallel.
The criteria for a good tool in this area is mostly how easy to write code it is with the tool and how maintainable the code is, and of course how fast the eventual program is.

\paragraph{Comparison breakdown}

\begin{table}[H]
  \begin{center}
    \begin{tabular}{ | l | p{10cm} | }
      \hline
      Technology & Description  \\ \hline

      POSIX Threads \cite{posix-threads} &
      \begin{itemize}
        \item Defines a set of C types, functions and constants 
        \item Spawn concurrent units of processing
        \item Achieve big speedups, as all cores of CPU are used at the same time.
        \end{itemize}\\ \hline

      OpenMP \cite{Openmp} &
      \begin{itemize}
        \item An API that implements a multi-thread, shared memory form of parallelism.
        \item Uses a set of compiler directives 
        \item Take care of many of low-level details
      \end{itemize}\\ \hline

      MPI \cite{mpi} &
      \begin{itemize}
        \item Core syntax and semantics of library
        \item Complexity, scope and control
        \item Manage allocation, communication, and synchronization of a set of processes 
      \end{itemize}\\ \hline
    \end{tabular}
  \end{center}
  \caption{Technology evaluated to perform parallel computation of XML transformations}
\end{table}

\paragraph{Discussion}

Pthreads is a standard for prgramming with threads, it can achieve big speedups, as all cores of your CPU are used at the same time.
OpenMP uses a set of compiler directives that are incorporated at compile-time to generate a multi-threaded version of your code.
MPI allows us to manage allocation, communication, and synchronization of a set of process.

\paragraph{Selection}

We will use MPI because it is a high-level standard, and so it will be easy to develop with.

% Section Six: Web API
% Owned by Elijah C. Voigt
\subsubsection{Web API (Elijah C. Voigt)}

\paragraph{Options}

Our web API may be implemented via a C++ web application, a Python or Ruby web application, or an Apache webserver CGI script.
Each of these has pros and cons, and each get the job done at some cost and with some benefits.

\paragraph{Goals for use in design}

XZES40-Transformer will be accessible via a web API.
This can be implemented a few different ways, but all of them must accomplish the same goal of allowing people to use the service over a network.
The three options being evaluated here vary in how they achieve this goal, and so they represent more their technology and less the specific implementation which will be used.

\paragraph{Criteria being evaluated}

The core of this application is related to document transformation.
The more time that is spent on non-document transformation tasks should be kept to a minimum.
Any technology we use to implement our web API should be simple, easy to develop, and easy to maintain.
In short, \textit{keep it simple stupid}.

\paragraph{Comparison breakdown}

Our first option is to use an Apache CGI script, which would be a simple Python, Perl, or Ruby file which calls our C program and returns the results (transformed document or error) to the user, all via an Apache server gateway.
The second options is to write a native web application using a web-app framework like Kore to handle HTTP requests.
The third option is to use a python framework like Flask to handle HTTP requests.
Each of these would be something that calls our document transformation app and exposes it to the outside world.
How we handle that is important to consider.

\begin{center}
  \begin{tabular}{ | l | p{10cm} |}
    \hline
    Technology & Description  \\ \hline
    Apache CGI Script \cite{cgi-tutorial} &
    \begin{itemize}
      \item HTTP requests are handled by the Apache web-server.
      \item XZES functionality is called by ``shelling out'' to the program and returning the results.
      \item Apache CGI Script requires a running Apache Server on the host.
    \end{itemize}\\ \hline
    Kore web-app framework \cite{kore-io} \cite{kore-feature} &
    \begin{itemize}
      \item HTTP Requests are handled by the Kore framework.
      \item XZES functionality is called natively with C code.
      \item Kore web-app framework acts as an independent daemon.
    \end{itemize}\\ \hline
    Flask web-app framework \cite{flask-site} &
    \begin{itemize}
      \item HTTP Requests are handled by the Flask framework.
      \item XZES functionality is called either natively or by using \inlinecode{exec} to ``shell out''.
      \item Flask web-app framework can act as an independent daemon or be managed by Apache.
    \end{itemize}\\ \hline
  \end{tabular}
\end{center}

\paragraph{Discussion}

The above three technologies are all entirely valid choices for our application; each approach the problem from different angles.

The Apache CGI choice is the Occam's Razor solution relative to the others.
Using the Apache web server we can register a script (written in Python, Perl, Ruby, etc) to accept requests and return a response.
This is simple, maintainable, and easy to create; this is especially appealing if we are only concerned with implementing the API and not fancy features like accessing a database or storing user sessions.
This option also allows us to leverage existing Apache Web-server power like load balancing and simple authentication without needing to write those features ourselves, they're just an Apache configuration file away from being a reality.

Kore is a web-app framework which would allow us to develop our application in C/C++, which has it's pros and cons.
C/C++ is notoriously difficult to write, and harder to write \textit{well}, so it may be a time-sink.
That said, it is nice to have a project which is written entirely in one language as contributors (ourselves and others) do not need to learn multiple languages to contribute to the transformer project.

Flask is another web-app framework, but one which is substantially easier to read and write.
This has the benefit of being easier to maintain than a Kore framework, and we can write exactly the level of complexity we want from our API.
On the other hand we would need to maintain a knowledge base of python, python frameworks, and python dependencies, so this is versatile but ultimately not necessarily easier to maintain than a kore framework.

\paragraph{Selection}

Since we are working with Apache on this project, we want to develop a simple solution to our API problem, and the Apache Web server is a powerful tool we will choose to use this in our design.
We will write a simple CGI script (which calls our C binary) and hook this into an Apache Web server.
We will need to depend on the Apache Web server for our project's package, but this should not be as hard as writing a web app ourselves.
We can also use the server to easily host our Web UI, which is a nice bonus.

% Section Ten: CLI
% Owned by Elijah C. Voigt
\subsubsection{Command Line Interface (Elijah C. Voigt)}

\paragraph{Options}

For the CLI we can develop a native C client, a simple Bash client, or split the difference with a Python client.

\paragraph{Goals for use in design}

Each of these CLI tools must be able to construct an HTTP POST request with two documents, send that request to a server, and parse a response payload.

\paragraph{Criteria being evaluated}

The criteria for this, as it is not part of the core functionality, is simplicity to write, maintain, and use for end-users.

\paragraph{Comparison breakdown}

\begin{center}
  \begin{tabular}{ | l | p{10cm} |}
    \hline
    Technology & Description  \\ \hline
    C/C++ \cite{} &
    \begin{itemize}
      \item Keeps the code base exclusively C/C++.
      \item Can be distributed with system package managers.
      \item Difficult to write and maintain.
    \end{itemize}\\ \hline
    Python \cite{} &
    \begin{itemize}
      \item Simple to write and maintain.
      \item Requires few external dependencies.
      \item Can be distributed with the \inlinecode{pip} package manager.
    \end{itemize}\\ \hline
    Bash \cite{} &
    \begin{itemize}
      \item Very simple to write.
      \item Quick to write using existing system tools like \inlinecode{curl} or \inlinecode{wget}.
    \end{itemize}\\ \hline
  \end{tabular}
\end{center}

\paragraph{Discussion}

The C/C++ option is not necessary at all.
This is an option which ought to be considered, but ultimately isn't worth pursuing as it will be very complicated to write and maintain, especially when simpler script-based options exist.

Python is a great middle ground.
The language comes with many web-request libraries and provides tools for users to upload their application to the \inlinecode{pip} package manager.
This means we can write a tool which performs well, is easy to maintain, requires few external dependencies, and can be downloaded by a package manager.
It will also be available on all platforms which run python, which includes Unix and Windows systems.

Bash is a viable candidate, especially when other tools like \inlinecode{curl} and \inlinecode{wget} will make quick work of the task.
The downside to this is that we cannot host our CLI on any standard package manager for verifiable distribution.
This forces users to download the CLI from our website directly.
Ultimately this will probably be used as a proof of concept, but it is not a final product.

\paragraph{Selection}

We will start by creating a CLI in Bash for testing purposes, and if time allows we will create a more polished CLI in Python.
Python language offers the best of both worlds in terms of simplicity, maintainability, and ease of use for end users, but for the sake of ``getting something out the door'' we will first create a tool that works in Bash.

% Section Seven: Website
% Owned by Shuai Peng
\subsubsection{Website UI}

The Web Interface will be a simple webpage which calls our web API.
For this reason this section focuses mostly on design technologies and less on HTTP request-handling technologies.

\paragraph{Option}

This document reviews three possible technologies we can use to implement our website .
First is plain-text HTML, CSS and Javascript, second is the Bootstrap front-end framework, and third is the Foundation framework.

\paragraph{Goals for use in design}

XZES40-Transformer will have a web-based user interface, which will also be our main user interface.
In considering which technology to use we will focus on making it conform with modern website design practices.
This will allow us to write an application which is hopefully user friendly and intuitive to use.
The website should also be look good.

\paragraph{Criteria being evaluated}

The most important aspect to consider for this user interface is the appearance.
We want to create web pages that work well on most screen sizes.
So the cost, efficiency, visual appeal, and dynamic screen adjustment are what we consider most important in our technology of choice.

\paragraph{Comparison breakdown}

\begin{itemize}
  \item {
    Cost:
    All of our options are open source and totally free and offer free documentation / tutorials.
  }
  \item {
%    Efficiency: CSS give simple plain-text interface.  We don't actually know what the size is, and what the pixel will be on the screen.  We have to try it, then we know what it looks like on our screen.  However, bootstrap and foundation is responsive design.  Both of them can change the size automatically in different size of screen.Both of them provide templates for creating web pages, but CSS do not provide.
    Efficiency:
    Using plain CSS/Javascript/HTML will perform well on most end-user's web-browsers, however it will be difficult to optimize the website to be responsive and adjust for smaller screen sizes.
    Bootstrap and Foundation were created with responsive design in mind.
    Both of them can change the size automatically in different size of screen.
    Both of them provide templates for creating web pages, but plain CSS do not provide.
  }
  \item {
%     Learning speed: CSS is the basis of HTML style sheet.  It is easy to understand and learning, however it hard to make good design.  Bootstrap and foundation spends time to learn, but after we learn the basically knowledge, bootstrap and foundation will be faster than CSS.
    Learning speed:
    CSS is the basis of HTML style sheet.
    It is easy to understand and learn, however it hard to make it look good.
    Bootstrap and foundation spends time to learn, but after we learn the basically knowledge, bootstrap and foundation will be faster than CSS.
  }
\end{itemize}

\begin{center}
    \begin{tabular}{ | l | p{10cm} |}
      \hline
      Technology & Description  \\ \hline
      CSS \cite{CSS_intro}&
      \begin{itemize}
        \item CSS is open source, and it the basis style sheet for HTML.
        \item CSS is not able to easily create a web page that fit in different size of screen.
        \item CSS is easy to learn, but it hard to use to make a good website.
      \end{itemize} \\ \hline

      Bootstrap \cite{boot_intro}&
      \begin{itemize}
        \item Bootstrap is open source project with good forum support.
        \item Bootstrap is efficient because it has responsive deign. It provide many templates.
        \item Bootstrap is easy to learn and use.
      \end{itemize} \\ \hline

      Foundation \cite{foundation_intro}&
      \begin{itemize}
        \item Foundation is open source project.
        \item Foundation is efficiency, and it has responsive deign. It provide many templates.
        \item Foundation is easy to lean and provides free tutorials.
      \end{itemize} \\ \hline

    \end{tabular}
\end{center}

\paragraph{Discussion}

% The table show us that all of them did similar work. However, they have different advantages and drawback. CSS is basis of HTML style sheet, and it works great, but it hard to create web pages beautiful. CSS is not responsive design. This make us hard to move web page into different size of screen, so we don't want to take CSS as our solution technology. Bootstrap and foundation does the same work. Both of them is open source and free to use. However bootstrap is much more stable and more templates, and there is free instructions in the W3C school. Foundation provides the tutorials, but it need to pay.
The table show us that all of them did similar work however, they have different advantages and drawback.
CSS is basis of HTML style sheet, and it works great, but it hard to create web pages beautiful using \textit{just} CSS and HTML.
This make it hard to move web page into different size of screen, so we don't want to take CSS as our solution technology.

Bootstrap and foundation do similar work, and both are open source and free to use.
However bootstrap is much more stable and provides more templates, and there is free instructions in the W3C school for using it.

\paragraph{Selection}

The best option is the Bootstrap for our project because Bootstrap is open source project and has good tutorials in W3C schools.
Bootstrap is the most popular font-end framework for web design, and it still updated by thousands of people.
During we are developing our code, I think that we should have second choice.
The Second choice is plain-HTML interface.
Althoght plain-HTML dose not look beatiful, it is most straight way to present information to user.
Keeping simple is the best way.

% Section Eight: Debian Package
% Owned by Zixun Lu 
%\subsubsection{Debian Package}
%XZES40-Transformer target the Debian operating system. Our team will upload XZES-40.deb through libraries(Xerces, Xalan) to the Apache website. The users can directly download from the website and run in their Debian operation system. 
%\paragraph{Binary packages}
%Binary packages, which contain executables, configuration files, man/info pages, copyright information, and other documentation. These packages are distributed in a Debian-specific archive format. They are usually characterized by having a '.deb' file extension. Binary packages can be unpacked using the Debian utility dpkg (possibly via a frontend like aptitude); details are given in its manual page.
%\paragraph{Source packages}
%Source packages, which consist of a .dsc file describing the source package (including the names of the following files), a .orig.tar.gz file that contains the original unmodified source in gzip-compressed tar format and usually a .diff.gz file that contains the Debian-specific changes to the original source. The utility dpkg-source packs and unpacks Debian source archives; details are provided in its manual page. (The program apt-get can be used as a frontend for dpkg-source.)
%\paragraph{}
\subsubsection{Debian \& Centos Package}

\paragraph{Options}


\paragraph{Goals for use in design}

Our team will release Debian and Centos packages.
Users can directly download these from the website and use these packages to install XZES40-Transformer on their Debian and Centos operation system.

\paragraph{Criteria being evaluated}

We may use Centos and Debian tools to build our package.
We can use Centos operation system to build Centos packages and use Debian operation system to build Debian packages.
Those tools are free to use.

\paragraph{Comparison breakdown}

\begin{center}
  \begin{tabular}{ | l | p{10cm} |}
    \hline
    Technology & Description  \\ \hline
    Centos packaging tools \cite{centos-tool} &
    \begin{itemize}
      \item Use in Centos operate system 
      \item It is esay to use
      \item Free.
    \end{itemize}\\ \hline
    FPM \cite{fpm-home} &
    \begin{itemize}
      \item Translates packages from one format to another.
      \item Allows re-use of other system's packages.
      \item Free.
    \end{itemize}\\ \hline
    Debian packaging tools \cite{debian-tool} &
    \begin{itemize}
      \item Use in Debian operate system
      \item It is easy to use
      \item Free.
    \end{itemize}\\ \hline
  \end{tabular}
\end{center}

\paragraph{Discussion}

The above tools are all valuable.
If we were going to just develop a Debian package we may only use the Debian tools, and the same goes for CentOS.
These tools are good at creating packages for those specific platforms, but since we intend to develop tools for mutiple platforms (Linux and BSD) using FPM to create cross-OS packages would be very convenient.

\paragraph{Selection}

In the end we will use FPM to develop our packages becuase it makes life very convenient.

% Section Ten: BSD Package
% Owned by Elijah C. Voigt
\subsubsection{BSD Package (Elijah C. Voigt)}

\paragraph{Options}

XZES40 Transformer will be created with an installation package so that it can easily be installed on a host system.
One platform we will create a package is FreeBSD.
This package will be a stretch goal.
This can be created manually, with FreeBSD system tools, or with a more general system-package creation tool.

\paragraph{Goals for use in design}

In an ideal world this tool would be easy to use, and would automatically take our source code, compile it, and package it for distribution.
Unfortunately that isn't a feasible solution for such a small project, but in our design we will be considering anything that gets us close to that reality.

\paragraph{Criteria being evaluated}

The technology chosen should be easy to use, and allow for modifications to the host package in an expedient way.
Creating system packages can be a pain, but creating one \textit{incorrectly} is even worse.
The tool chosen should be free, relatively painless to use, and require minimal human interaction to prevent the introduction of human errors.

\paragraph{Comparison breakdown}

\begin{table}[H]
  \begin{center}
    \begin{tabular}{ | l | p{10cm} |}
      \hline
      Technology & Description  \\ \hline
      Poudriere \cite{poudriere-tutorial} &
      \begin{itemize}
        \item Creates builds for multiple platforms including other versions of FreeBSD and other CPU architectures.
        \item Builds packages in parallel.
        \item Free.
      \end{itemize}\\ \hline
      FPM \cite{fpm-home} &
      \begin{itemize}
        \item Translates packages from one format to another.
        \item Allows re-use of other system's packages.
        \item Free.
      \end{itemize}\\ \hline
      pkg-create \cite{pkg-create-man} &
      \begin{itemize}
        \item Simple to use.
        \item Installed on most FreeBSD systems.
        \item Free.
      \end{itemize}\\ \hline
    \end{tabular}
  \end{center}
  \caption{Tools considered for the use of system package creation.}
\end{table}

\paragraph{Discussion}

Each of these three technologies offers similar end results, but some offer a better development experience or a more feature rich pipline.

Poudriere is appealing because it is by far the most feature rich option available.
It essentially offers a system for us to test and build our software all in one tool using the BSD Jails system.
The consequence is that this package would need to be created manually each time we want to release an update to our software, or just to make changes to the package.

FPM is appealing mostly because this FreeBSD package will be a stretch goal.
FPM will be able to take a Debian or CentOS package \textit{as well as} produce a FreeBSD package.
This will require minimal intervention, and as long as we can test that the package produced is correct we can use this for future iterations of the package.
The downside is that this is not guaranteed to work correctly as it does not provide any testing infrastructure for the package produced, so some amount of manual testing will be required.

The last option is not appealing, but viable all the same.
pkg-create can be used to create a package on FreeBSD.
The tool is convenient to acquire, and allows us to create the package, but does not offer nearly as many benefits as the others.

\paragraph{Selection}

For convenience we will use FPM.
Once we have a Debian package creating a FreeBSD package should be smooth sailing.
The solution is free, requires minimal human interaction, and allows us to create reproducible results (i.e., weather the package works or not is not determined by the package creator's coffee intake).

% Section Ten: Windows Package
% Owned by Shuai Peng
\subsubsection{Windows Package}

\paragraph{Option}

There is three option technology for the windows install package.
The first one is WiX Toolset, the second is EMCO MSI Package Builder, and the third is Advance Installer.

\paragraph{Goals for use in design}

XZES40-Transformer may be packaged to run on multiple platforms.
We want to create install package, so users do not need to install our package manually from source.

\paragraph{Criteria being evaluated}

Although windows install package is our option work for our project, we want it move from other platform smoothly.
We want it easy to learn how to use this technology, and we also want it free.
The flow of install should be simple for user, so user can just install and run application quickly.

\paragraph{Comparison breakdown}

\begin{itemize}
  \item {
    %      Cost: WiX Toolset is open source and free software. However, Advanced installer is company tools that we need to pay. EMCO MSI Package Builder provide free version for individual developer, but free version of package builder limit the functionality for create MSI package.
    Cost: WiX Toolset is open source and free software however the advanced installer is a proprietary tool which must be paid for.
    EMCO MSI Package Builder provide free version for individual developer, but free version of package builder limit the functionality for create MSI package.
  }
  \item {
    %Security: WiX Toolset is open source project since 2004. Even the Microsoft also uses WiX to create MSI package, so WiX tools should be the most safe tools than other tools. The security of Advanced installer is acceptable. The company is created since 2002, and other big company used this tools, such as Sony, Dell, etc. EMCO MSI Package Builder is same as the advance installer.
     Security: WiX Toolset is open source project since 2004.
     Even the Microsoft also uses WiX to create MSI package, so WiX tools should be the most safe tools than other tools.
     The security of Advanced installer is acceptable.
     The company is created since 2002, and other big company used this tools, such as Sony, Dell, etc.
     EMCO MSI Package Builder is same as the advance installer.
  }
  \item {
    %Stable: WiX Toolset is open source project, and there is thousand of developer contribute on this project, so WiX Toolset is more stale than other tools.
    Stable: WiX Toolset is open source project, and there is thousand of developer contribute on this project, so WiX Toolset is more stable than other open source alternatives.
  }
  \item {
    %Learning Speed: WiX Toolset have steep learning curve. It is better to understand the fact of Microsoft install package before using this tools. Advanced installer and EMCO MSI Package Builder provides GUI for user, it is easy to using and learning.
    Learning Speed: WiX Toolset has a steep learning curve.
    It is better to understand the fact of Microsoft install package before using this tools.
    Advanced installer and EMCO MSI Package Builder provides GUI for user, it is easy to using and learning.
  }
\end{itemize}
	
\begin{table}[H]
  \begin{center}
      \begin{tabular}{ | l | p{10cm} |}
        \hline
        Technology & Description  \\ \hline
        WiX Toolset \cite{Wix_tool} &
        \begin{itemize}
          \item WiX Toolset is open source project 
          \item WiX Toolset is more stable and security than other tools
          \item WiX Toolset have steed learning curve, but it is worth to learn
          \item WiX	Toolset is free for every one.
        \end{itemize} \\ \hline

        EMCO MSI Package Builder \cite{EMCO_MSI} &
        \begin{itemize}
          \item EMCO MSI Package Builder is not open source.
          \item EMCO MSI Package Builder is stable, and security is acceptable.
          \item EMCO MSI Package Builder is easy to learn, because it have GUI.
          \item EMCO MSI Package Builder has free version for individual developer.
        \end{itemize} \\ \hline

        Advance Installer \cite{advanced_install} &
        \begin{itemize}
          \item Advance Installer is close source tools.
          \item Advance Installer is stable and security.
          \item Advance Installer is easy to learn. It has GUI, and forum support.
          \item Advance Installer is expensive. No free version.
        \end{itemize} \\ \hline
      \end{tabular}
  \end{center}
  \caption{Technology evaluations for system packages.}
\end{table}

\paragraph{Discussion}

% WiX Toolset is a tools that build Windows installation packages from XML source code. Traditional setup tools used a programmatic, script-based approach to be installed on the target machine. However, WiX uses a different way. It is like a programming language, and it used a text file, which is based on the XML format, to describe all the steps of the installation process. Microsoft also uses WiX with all its major software packages. For example, the setup of Microsoft Office was developed entirely with WiX. \cite{Wix_tool}

The WiX toolset is a collection of tools which build Windows installation packages from XML source code.
Traditional setup tools used a programmatic, script-based approach to be installed on the target machine but WiX uses a different method.
It is uses an XML configuration file to describe the steps of the installation process.
Microsoft also uses WiX with all its major software packages, like Microsoft Office.
\cite{Wix_tool}

% EMCO MSI Package Builder is an installation tools designed for help developer create, maintain and distribute Windows Installer packages. \cite{EMCO_MSI} It helps developer create MSI package automatically by using changes tracking technology, which is used to generate installation project data, or manually by using the visual editor.

EMCO MSI Package Builder is an installation tools designed for help developer create, maintain and distribute Windows Installer packages.
\cite{EMCO_MSI}
It helps developer create MSI package automatically by using changes tracking technology, which is used to generate installation project data, or manually by using the visual editor.

% Advanced Installer is GUI tool that can simply create Microsoft Install packages. Advanced Installer create and maintain installation package, such as EXE, MSI, ETC. It based on the Windows Installer technology. \cite{advanced_install}

Advanced Installer is GUI tool that can simply create Microsoft Install packages.
Advanced Installer create and maintain installation package, such as EXE, MSI, ETC.
It based on the Windows Installer technology.
\cite{advanced_install}

\paragraph{Selection}

% After I compare all of above method. I decide to choose WiX as my solution technology. The first reason why WiX is my solution technology is that it is open source software. It's powerful set of tools available to create our Windows installation experience.
% The second reason is that it represents by source code rather than GUI. GUI maybe easy for developer, but it also hided every thing behind the GUI. If we get some wired problem, we can't solve it via GUI software. 
% Third reason is that it complete integration into application build processes. When install progress start, other setup modification are made in parallel, so no vital information will be lost. The setup program will complete together with the application itself. This is not required by client, but if client want to different way to create windows install package, we can choose advanced installer as our alternate technology.

After comparing all of above methods we have chosen to use WiX as the solution to our Windows packaging problem.
The first reason why WiX is my solution technology is that it is open source software.
It a powerful set of tools available to create our Windows installation package.

The second reason is that it represents by source code rather than GUI.
GUI may be easy for developer, but it also hided every thing behind the GUI.
If we get some weired problem, we can't solve it via GUI software.

Third reason is that it can be completely integrated into our application build processes.
When install progress start, other setup modification are made in parallel, so no vital information will be lost.

The fourth reason is that it use XML as the main language.
We don't need spend time on learning how to use WIX.

The setup program will complete together with the application itself.
This is not required by client, but if client want to different way to create windows install package, we can choose advanced installer as our alternate technology.

\subsubsection{Conclusion}

To summarize our findings:

\begin{center}
    \begin{tabular}{ | l | p{5cm} | r | }
    \hline
    Section & Author & Technology Choice\\ \hline
    Research and Benchmarking & Zixun Lu & ? \\ \hline
    XML/XSLT Document Transformation & Shuai Peng & ? \\ \hline
    XML/XSLT Document Parsing & Elijah C. Voigt & Xalan C++ \\ \hline
    XML/XSLT Document Caching & Shuai Peng & ? \\ \hline
    XML/XSLT Document Parallel computation & Zixun Lu & ? \\ \hline
    Web API & Elijah C. Voigt & Apache + Python CGI script \\ \hline
    Web Interface & Shuai Peng & ? \\ \hline
    Linux Package & Zixun Lu & ? \\ \hline
    Command Line Interface& Elijah C. Voigt & Bash proof of concept, Python final product \\ \hline
    Windows Package & Shuai Peng & ? \\ \hline
    BSD Package & Elijah C. Voigt & FPM package creation toolkit \\ \hline
    \end{tabular}
\end{center}

\clearpage


\subsection{Changes to technologies used}
%%%%%%%%%%%%%%%%%%%%%%%%%%%%%%%%%%%%%%%%%%%%%%%%%%%%%%%%%%%%%%%%%%%%%%%%%%%%%%%%
% Did you change your mind about any technologies?
% What had to change?
%%%%%%%%%%%%%%%%%%%%%%%%%%%%%%%%%%%%%%%%%%%%%%%%%%%%%%%%%%%%%%%%%%%%%%%%%%%%%%%%

\subsubsection{Interface (web) technologies}

In the course of development we did not do a particularly good job of defining how we ought to create the web front-end.
As a result it was thrown together fairly quickly a few weeks before the final deadline.
The tools used in this component were not explicitly researched, but they were well thought out before development.

Specifically we used jQuery to perform asynchronous calls to the Apache CGI script, FileSaver.js and Blob.js (discussed further in the project's Markdown documentation) to download the resulting files to the users computer, and raw HTML5 and CSS3 to format the page.
These technologies worked well and are industry standard tools for small front-end web pages like this one.

\subsubsection{Cache}

The cache ended up using a Key-value library provided to us by our client already in the Xerces/Xalan code-bases.
This was well-defined early on because we had not yet decided if we were going to write a cache system in-memory ourselves or use a different tool like Redis.
The key-value scheme worked well and was very performed well in testing.

\subsubsection{Parallel/Daemon}

Although we discussed using higher levels of abstraction for creating forking processes.
We ended up just using the system default Linux POSIX threads as this was the technology we were most comfortable with.

\subsubsection{Vagrant}

We expected to use VirtualBox for development but ended up using Vagrant as this was the tool one of the developers was most comfortable with and he was willing to put in extra effort to make this system robust enough for the project's needs.
This ended up helping the team write setup scripts, used by Vagrant to setup a "one command" testing environment, which can be used at a later date for creating a system package.

The Vagrant environment automatically downloads the required operating system (Debian), installs the required packages, copies over the required configuration files (in Apache and systemd), and starts the web server.
Users who want to contribute can run the service locally just need to install Vagrant and run `vagrant up' to get a development environment; further change are automatically copied into the VM and they can compile code in the os by running `vagrant ssh' and running `make' in the VM.

This tool was used for its robust features, simple interface, and cross platform compatibility.


\section{Development journal}
%%%%%%%%%%%%%%%%%%%%%%%%%%%%%%%%%%%%%%%%%%%%%%%%%%%%%%%%%%%%%%%%%%%%%%%%%%%%%%%%
% Introduction; what these journals were, how they were made, why they were made.
%%%%%%%%%%%%%%%%%%%%%%%%%%%%%%%%%%%%%%%%%%%%%%%%%%%%%%%%%%%%%%%%%%%%%%%%%%%%%%%%

\subsection{Zixun Lu}

% \paragraph{WEEK #; DATE:}

\subsection{Shuai Peng}

% \paragraph{WEEK #; DATE:}

\subsection{Elijah C. Voigt}

% \paragraph{WEEK #; DATE:}


\section{Project documentation}
%%%%%%%%%%%%%%%%%%%%%%%%%%%%%%%%%%%%%%%%%%%%%%%%%%%%%%%%%%%%%%%%%%%%%%%%%%%%%%%%
% Project documentation.
% > How does your project work?
% > > What is its structure?
% > > What is its Theory of Operation?
% > > Block and flow diagrams are good here.
% > How does one install your software, if any?
% > How does one run it?
% > Are there any special hardware, OS, or runtime requirements to run your software?
% > Any user guides, API documentation, etc.
% This needs to be detailed enough to recreate and/or use your project! 
%%%%%%%%%%%%%%%%%%%%%%%%%%%%%%%%%%%%%%%%%%%%%%%%%%%%%%%%%%%%%%%%%%%%%%%%%%%%%%%%
% Short introduction about what the docs were, how they were created, etc.
%%%%%%%%%%%%%%%%%%%%%%%%%%%%%%%%%%%%%%%%%%%%%%%%%%%%%%%%%%%%%%%%%%%%%%%%%%%%%%%%

% \subsection{DOCS 1}
% \subsection{DOCS 2}
% \subsection{...}


\section{Learning: Technical}
%%%%%%%%%%%%%%%%%%%%%%%%%%%%%%%%%%%%%%%%%%%%%%%%%%%%%%%%%%%%%%%%%%%%%%%%%%%%%%%%
% How did you learn new technology?
% > What web sites were helpful? (Listed in order of helpfulness.)
% > What, if any, reference books really helped?
% > Were there any people on campus that were really helpful?
%%%%%%%%%%%%%%%%%%%%%%%%%%%%%%%%%%%%%%%%%%%%%%%%%%%%%%%%%%%%%%%%%%%%%%%%%%%%%%%%


\section{Learning: Personal}
%%%%%%%%%%%%%%%%%%%%%%%%%%%%%%%%%%%%%%%%%%%%%%%%%%%%%%%%%%%%%%%%%%%%%%%%%%%%%%%%
% What did you learn from all this?
% One per team member.
% > What technical information did you learn?
% > What non-technical information did you learn?
% > What have you learned about project work?
% > What have you learned about project management?
% > What have you learned about working in teams?
% > If you could do it all over, what would you do differently?
% Be honest here -- no B.S.
%%%%%%%%%%%%%%%%%%%%%%%%%%%%%%%%%%%%%%%%%%%%%%%%%%%%%%%%%%%%%%%%%%%%%%%%%%%%%%%%

%%%%%%%%%%%%%%%%%%%%%%%%%%%%%%%%%%%%%%%%%%%%%%%%%%%%%%%%%%%%%%%%%%%%%%%%%%%%%%%%
% What did you learn from all this?
% One per team member.
% > What technical information did you learn?
% > What non-technical information did you learn?
% > What have you learned about project work?
% > What have you learned about project management?
% > What have you learned about working in teams?
% > If you could do it all over, what would you do differently?
% Be honest here -- no B.S.
%%%%%%%%%%%%%%%%%%%%%%%%%%%%%%%%%%%%%%%%%%%%%%%%%%%%%%%%%%%%%%%%%%%%%%%%%%%%%%%%

\section{Zixun Lu}


%%%%%%%%%%%%%%%%%%%%%%%%%%%%%%%%%%%%%%%%%%%%%%%%%%%%%%%%%%%%%%%%%%%%%%%%%%%%%%%%
% What did you learn from all this?
% One per team member.
% > What technical information did you learn?
% > What non-technical information did you learn?
% > What have you learned about project work?
% > What have you learned about project management?
% > What have you learned about working in teams?
% > If you could do it all over, what would you do differently?
% Be honest here -- no B.S.
%%%%%%%%%%%%%%%%%%%%%%%%%%%%%%%%%%%%%%%%%%%%%%%%%%%%%%%%%%%%%%%%%%%%%%%%%%%%%%%%

\section{Shuai Peng}


%%%%%%%%%%%%%%%%%%%%%%%%%%%%%%%%%%%%%%%%%%%%%%%%%%%%%%%%%%%%%%%%%%%%%%%%%%%%%%%%
% What did you learn from all this?
% One per team member.
% > What technical information did you learn?
% > What non-technical information did you learn?
% > What have you learned about project work?
% > What have you learned about project management?
% > What have you learned about working in teams?
% > If you could do it all over, what would you do differently?
% Be honest here -- no B.S.
%%%%%%%%%%%%%%%%%%%%%%%%%%%%%%%%%%%%%%%%%%%%%%%%%%%%%%%%%%%%%%%%%%%%%%%%%%%%%%%%

\section{Elijah C. Voigt}




\subsection{Expo poster}
%%%%%%%%%%%%%%%%%%%%%%%%%%%%%%%%%%%%%%%%%%%%%%%%%%%%%%%%%%%%%%%%%%%%%%%%%%%%%%%%
% Your final poster, scaled and color-printed on a single 8.5"x11" paper.
% If you don't have access to a color printer, I will print it for you.
% Let me know by the Thursday of dead week if you need it printed.
%%%%%%%%%%%%%%%%%%%%%%%%%%%%%%%%%%%%%%%%%%%%%%%%%%%%%%%%%%%%%%%%%%%%%%%%%%%%%%%%
% Brief introduction about what the poster was and what it was for.
%%%%%%%%%%%%%%%%%%%%%%%%%%%%%%%%%%%%%%%%%%%%%%%%%%%%%%%%%%%%%%%%%%%%%%%%%%%%%%%%


\section{Appendix 1}
%%%%%%%%%%%%%%%%%%%%%%%%%%%%%%%%%%%%%%%%%%%%%%%%%%%%%%%%%%%%%%%%%%%%%%%%%%%%%%%%
% % Appendix 1: Essential Code Listings.
% You don't have to include absolutely everything, but if someone wants to understand your project, there should be enough here to learn from.
% If you worked within a larger project, something like a patch file might be a good way to go.
%%%%%%%%%%%%%%%%%%%%%%%%%%%%%%%%%%%%%%%%%%%%%%%%%%%%%%%%%%%%%%%%%%%%%%%%%%%%%%%%


\section{Appendix 2}
%%%%%%%%%%%%%%%%%%%%%%%%%%%%%%%%%%%%%%%%%%%%%%%%%%%%%%%%%%%%%%%%%%%%%%%%%%%%%%%%
% Appendix 2: Anything else you want to include.
% Photos, etc.
%%%%%%%%%%%%%%%%%%%%%%%%%%%%%%%%%%%%%%%%%%%%%%%%%%%%%%%%%%%%%%%%%%%%%%%%%%%%%%%%


\end{document}
